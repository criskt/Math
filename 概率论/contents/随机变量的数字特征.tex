\section{随机变量的数字特征}
\subsection{随机变量的数学期望和方差}
\begin{enumerate}
	\item 数学期望
	\begin{enumerate}
		\item 离散型
			\begin{equation}
				E(X) = \sum_{k=1}^{+\infty}x_kp_k
			\end{equation}
			如果上式绝对收敛,则期望存在;否则期望不存在
		\item 连续型
			\begin{equation}
				E(X) = \int_{-\infty}^{+\infty}xf(x)dx
			\end{equation}
			同样地,若上式绝对收敛,则期望存在;否则不存在
	\end{enumerate}

	\item 数学期望的性质
	\begin{enumerate}
		\item 
		\begin{equation}
			E(CX) = CE(X)
		\end{equation}
		\item 
		\begin{equation}
			E(X \pm Y) = E(X) \pm E(Y)
		\end{equation}
		\item 若随机变量$X$, $Y$相互独立,则
		\begin{equation}
			E(XY) = E(X)E(Y)
		\end{equation}
		其实,$X, Y$不相关即可,不相关就是上式成立的充要条件
	\end{enumerate}

	\item $Y=g(X)$的数学期望
	\begin{enumerate}
		\item 离散型
			\begin{equation}
				E(Y) = E(g(X)) = \sum_{k=1}^{+\infty} g(x_k)p_k
			\end{equation}
			如果上式绝对收敛,则期望存在;否则期望不存在
		\item 连续型
			\begin{equation}
				E(Y) = E(g(X)) = \int_{-\infty}^{+\infty}g(x)f(x)dx
			\end{equation}
			同样地,若上式绝对收敛,则期望存在;否则不存在
	\end{enumerate}

	\item $Z=g(X,Y)$的数学期望
	\begin{enumerate}
		\item 离散型
			\begin{equation}
				E(Y) = E(g(X,Y)) = \sum_{i=1}^{+\infty}\sum_{j=1}^{+\infty} g(x_i, y_j)p_{ij}
			\end{equation}
			如果上式绝对收敛,则期望存在;否则期望不存在
		\item 连续型
			\begin{equation}
				E(Y) = E(g(X,Y)) = \int_{-\infty}^{+\infty}\int_{-\infty}^{+\infty}g(x,y)f(x,y)dxdy
			\end{equation}
			同样地,若上式绝对收敛,则期望存在;否则不存在
	\end{enumerate}

	\item 方差
	\begin{enumerate}
		\item 方差的定义
		\begin{equation}
			D(X) = E\{\left[X-E(X)\right]^2\}
		\end{equation}
		记做$\sigma^2$
		\item 方差计算公式
		\begin{equation}
			D(X) = E(X^2) - \left[ E(X) \right]^2
		\end{equation}
		另,因为方差恒$\geq 0$,故由上式可得出$E(X^2) \geq \left[ E(X) \right]^2$
	\end{enumerate}

	\item 方差的性质
	\begin{enumerate}
		\item 
		\begin{equation}
			D(aX+b) = a^2D(X)
		\end{equation}
		\item $X,Y$相互独立时
		\begin{equation}
			D(X\pm Y) = D(X) + D(Y)
		\end{equation}
		注意,不论是$X+Y$还是$X-Y$,最后的结果均是方差之和
	\end{enumerate}

	\item 常用随机变量的数学期望和方差
	\begin{enumerate}
		\item $0-1$分布
		\begin{align}
			E(X) &= p \\
			D(X) &= p(1-p)
		\end{align}
		\item 二项分布: $X \sim B(n,p)$
		\begin{align}
			E(X) &= np \\
			D(X) &= np(1-p)
		\end{align}
		\item 泊松分布: $X \sim P(\lambda)$
		\begin{align}
			E(X) &= \lambda \\
			D(X) &= \lambda
		\end{align}
		\item 几何分布: $P\{X=k\} = p(1-p)^{k-1}, \quad k = 1, 2, \dots, \quad 0<p<1$
		\begin{align}
			E(X) &= \frac{1}{p} \\
			D(X) &= \frac{1-p}{p^2}
		\end{align}
		\item 均匀分布: $X \sim U(a,b)$
		\begin{align}
			E(X) &= \frac{a+b}{2} \\
			D(X) &= \frac{(b-a)^2}{12}
		\end{align}
		\item 指数分布: $X \sim E(\lambda)$
		\begin{align}
			E(X) &= \frac{1}{\lambda} \\
			D(X) &= \frac{1}{\lambda^2}
		\end{align}
		\item 正态分布: $X \sim N(\mu, \sigma^2)$
		\begin{align}
			E(X) &= \mu \\
			D(X) &= \sigma^2
		\end{align}
	\end{enumerate}
\end{enumerate}




















