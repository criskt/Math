\section{经验风险最小化}

\subsection{偏差\&方差}
\begin{enumerate}
	\item 高偏差(Bias)对应欠拟合
	\item 高方差(Variance)对应过拟合
\end{enumerate}

\subsection{经验风险最小化(ERM)}
\begin{enumerate}
	\item 定义数据集:
	\begin{align}
		S = \{x^{(i)}, y^{(i)}\}, \quad 1 \leq i \leq m
	\end{align}
	其中,$x^{(i)}, y^{(i)}$独立同分布(IID),$y \in \{0, 1\}$
	\item 分类模型为:
	\begin{align}
		h_{\theta}(x) &= g(\theta^Tx) \\
		\text{其中:}
		g(z) &= 1\{z\geq 0\}, \quad g \in \{0, 1\}
	\end{align}
	所以,$h_{\theta}(x)$只会是$0$或$1$
	\item 定义训练误差为:
	\begin{align}
		\hat{\varepsilon}(h_{\theta}) = \hat{\varepsilon_S}(h_{\theta}) = \frac{1}{m}\sum_{i=1}^{m}1\{h_{\theta}(x^{(i)}) \neq y^{(i)}\}
	\end{align}
	\item 经验误差最小化为:
	\begin{align}
		\hat{\theta} = \arg_{\theta}\min{\hat{\varepsilon_S}(h_{\theta})}
	\end{align}
	即选择使训练误差最小的参数$\theta$。\\
	对于ERM来说,因为他是非凸的,用一般算法不好优化。
	\item 我们再定义另一种等价的ERM,首先假设模型集合:
	\begin{align}
		\mathcal{H} = \{h_{\theta}: \theta \in R^{n\times 1}\}
	\end{align}
	其中,$h_{\theta}$为分类模型,输出为$0$或$1$。
	\item 其训练误差为:
	\begin{align}
		\hat{\varepsilon}(h) = \frac{1}{m}\sum_{i=1}^{m}1\{h(x^{(i)}) \neq y^{(i)}\}
	\end{align}
	\item 那么,ERM定义为:
	\begin{align}
		\hat{h} = \arg_{h \in \mathcal{H}} \min{\hat{\varepsilon}(h)}
	\end{align}
	\item 泛化能力定义为:
	\begin{align}
		\varepsilon(h) = P_{x,y \sim D} (h(x) \neq y)
	\end{align}
	接下来的任务是证明最优化ERM能带来较小的泛化误差,为此,我们先引入两个引理。
\end{enumerate}


\subsection{联合界 \& Hoeffding不等式}
\subsubsection{联合界定理}
\begin{enumerate}
	\item 令$A_1, A_2, \dots, A_k$是$k$个任意事件(即不要求独立,也不要求同分布,也没有其他要求),那么:
	\begin{align}
		P(A_1 \cup A_2 \cup \dots \cup A_k) \leq P(A_1) + P(A_2) + \dots + P(A_k)
	\end{align}
	此定理可画图证明,不赘述
\end{enumerate}

\subsubsection{Hoeffding不等式定理}
\begin{enumerate}
	\item 令$Z_1, Z-2, \dots, Z_m$为$m$个独立同分布(IID)变量,他们都服从Bernoulli分布,即:
	\begin{align}
		P(Z_i = 1) &= \varphi \\
		P(Z_i = 0) &= 1 - \varphi
	\end{align}
	我们使用这$m$个数据来估计$\varphi$,得:
	\begin{align}
		\hat{\varphi} = \frac{1}{m}\sum_{i=1}^{m}Z_i
	\end{align}
	那么,Hoeffding不等式即为:对于任意固定的数$\gamma > 0$,存在:
	\begin{align}
		P(|\hat{\varphi} - \varphi| > \gamma) < 2e^{-2\gamma^2 m}
	\end{align}
	这个定理的意义在于,当样本数目$m$增大时,我们对参数的估计将越来越接近真实值。
\end{enumerate}


\subsection{一致收敛定理}
\begin{enumerate}
	\item 定义模型集合为:
	\begin{align}
		\mathcal{H} = \{h_1, h_2, \dots, h_k\}
	\end{align}
	\item 一致收敛定理为:\\
	\begin{align}
		P(\forall h_i \in \mathcal{H}, |\varepsilon(h_i)-\hat{\varepsilon(h_i)}|>\gamma) \geq 1- 2ke^{-2\gamma^2 m}
	\end{align}
	该式即为一致收敛定理,它的意义在于,至少有$1- 2ke^{-2\gamma^2 m}$的概率使得:对模型集合中的所有假设,其泛华误差都在训练误差的$\gamma$范围内。
\end{enumerate}



\subsection{一致收敛的推论}
在一致收敛中,三个参数$m, \gamma$,以及概率$P$是相互关联的,我们可以通过固定其中的2个来推出第三个。
\begin{enumerate}
	\item $P(m, \gamma)$上面已经给出,其余2个如下,证明过程略
	\item  $m$ 与$\gamma, P$的关系
	\begin{align}
		m \geq \frac{1}{2\gamma^2}\log\frac{2k}{\sigma}
	\end{align}
	此式说明了一个算法或模型要达到一个确定的性能时,需要的样本数目。也称为算法的样本复杂度。
	\item $\gamma$与$m, P$的关系
	\begin{align}
		\gamma = \sqrt{\frac{1}{2m}\log\frac{2k}{\sigma}}
	\end{align}

	\item 在一致收敛成立的情况下,我们通过ERM方法得到的假设$\hat{h}$的泛华能力如何?
	\begin{enumerate}
		\item 定义:
		\begin{align}
			h^* = \arg_{h\in\mathcal{H}}\min \varepsilon(h)
		\end{align}
		其中,$h^*$为泛华误差最小的假设
		\item 可以推导出:
		\begin{align}
			\varepsilon(\hat{h}) \leq \hat{\varepsilon}(\hat{h}) + \gamma \leq \hat{\varepsilon}(h^*) + \gamma \leq \varepsilon(h^*) + \gamma + \gamma
			= \varepsilon(h^*) + 2\gamma
		\end{align}


	\end{enumerate}
	{\color{red}{其余内容略,有用了再看。}}
\end{enumerate}

\subsection{说明}
\begin{enumerate}
	\item 本部分内容来源:斯坦福大学《机器学习公开课》第9课:\href{http://open.163.com/movie/2008/1/F/H/M6SGF6VB4_M6SGJV3FH.html}{经验风险最小化}
	\item 本笔记内容参考:本文参考以下笔记:\href{http://blog.csdn.net/stdcoutzyx/article/details/12110337}{课程笔记参考}
\end{enumerate}



