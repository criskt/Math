% !TEX program = xelatex
% !TEX TS-program = xelatex
% !TEX encoding = UTF-8 Unicode

\documentclass[a4paper, 12pt]{article}
\usepackage{amsmath}
\usepackage{fontspec}  % 解决中文问题
	\defaultfontfeatures{Mapping=tex-text}
	\setromanfont{Songti SC}  % Mac下使用此句
	% \setromanfont{"[msyh.ttc]"}  % Windows下使用此句
	\XeTeXlinebreaklocale “zh”
	\XeTeXlinebreakskip = 0pt plus 1pt minus 0.1pt
\usepackage[top=20mm, bottom=25mm, left=20mm, right=25mm]{geometry}  % 设置页边距
	% \special{papersize=12cm,9cm}  % For Kindle
	\setlength\paperheight{35cm}  % For iPhone 7
	\setlength\paperwidth{20cm}
\usepackage{tikz}  % 画图用s
	\def\pgfsysdriver{pgfsys-dvipdfmx.def}
\usepackage{xltxtra}
\usepackage{xunicode}

\renewcommand{\thefootnote}{注[\arabic{footnote}]}  % 重新定义\footnote命令的格式





\begin{document}
	\title{机器学习相关基础}
\author{MingShun Wu}
\date{\today}
\maketitle
	\renewcommand{\contentsname}{目录}
\tableofcontents
\setcounter{tocdepth}{3}

	\part{CS299课程笔记}
	% 	\title{CS299课程笔记}
\author{MingShun Wu}
\date{\today}
\maketitle
		% \renewcommand{\contentsname}{目录}
\tableofcontents
\setcounter{tocdepth}{3}
		\section*{说明}
\begin{enumerate}
	\item 本文是根据吴恩达(Andrew Ng) CS229课程的视频\&讲义,再加上一些自己的理解整理出来的,我也只是个初学者,里面的内容可能有很多错误,若发现错误欢迎联系我修改: \href{mailto:wu.mingshun@icloud.com}{wu.mingshun@icloud.com}
	\item 暂时只有部分内容,通过后面的网址可以下载到最新的文档(若有更新会替换掉原来的文件),下载最新文档:\href{http://images.miasun.cn/math/机器学习基础知识.pdf}{http://images.miasun.cn/math/机器学习基础知识.pdf}
	% \item 
	% 个人联系方式:
	% \begin{enumerate}
	% 	\item 邮件:
	% 	\item 微信
	% 	\begin{figure}[htbp]
	% 		\centering
	% 		\includegraphics[scale=0.25]{images/微信号二维码}
	% 		\caption{微信号二维码}
	% 	\end{figure}
	% \end{enumerate}
	
\end{enumerate}





























		\section{线性回归}
\subsection{假设函数(Hypothesis Function)}
\begin{equation}\begin{aligned}
	h_\theta(x) = \theta_0 + \theta_1x_1 + \theta_2x_2 + \theta_3x_3 + ... + \theta_nx_n
\end{aligned}\end{equation}
为每个数据点添加$x_0=1$,则
\begin{equation}\begin{aligned}
	h_\theta(x) &= \theta_0x_0 + \theta_1x_1 + \theta_2x_2 + \theta_3x_3 + ... + \theta_nx_n \\
	&= \sum_{j=0}^n{\theta_jx_j}
\end{aligned}\end{equation}

\subsection{最小二乘法}
后面要讲到的Cost Function使用的就是最小二乘法

\subsection{梯度下降}
\subsubsection{批梯度下降(BGD)}
\begin{enumerate}
	\item Cost Function
	\begin{equation}\begin{aligned}
		J(\theta) = \frac{1}{2} \sum_{i=1}^m \left[h_{\theta} {(x^{(i)})} - y^{(i)}\right]^2
	\end{aligned}\end{equation}
	其中,$J(\theta)$就是在训练过程中的Cost Function。我们的目的就是将该Cost Function最小化。\\
	上式中:
	\begin{itemize}
		\item $x^{(i)}$为输入,是个$n$维列向量;
		\item $y^{(i)}$为输出,是一个标量;
		\item $\theta$为所要训练的参数,也是一个$n$维列向量;
		\item $h_{\theta}$为假设函数,我们通过它来预测$y^{(i)}$的值;
		\item $h_{\theta} {(x^{(i)})}$为输入经过假设函数后得到的输出,是$y^{(i)}$的点估计量
	\end{itemize}

	\item 迭代方式
	\begin{equation}\begin{aligned}
	      \theta_j &:= \theta_j - \alpha \frac{\partial} {\partial \theta_j} J(\theta)
	\end{aligned}\end{equation}
	其中:
	\begin{equation}\begin{aligned}
	      \frac{\partial} {\partial \theta_j} J(\theta) &= \frac{\partial}{\partial \theta_j} \frac{1}{2} \sum_{i=1}^m\left[ h_\theta(x^{(i)}) - y^{(i)} \right]^2 \\
	      &= 2 * \frac{1}{2} \sum_{i=1}^m\left[ h_\theta(x^{(i)}) - y^{(i)} \right] \frac{\partial}{\partial\theta_j}\left[ h_\theta(x^{(i)}) - y^{(i)} \right] \\
	      &= \sum_{i=1}^m\left[ h_\theta(x^{(i)}) - y^{(i)} \right]\frac{\partial}{\partial\theta_j}\left[ \theta_0x_0^{(i)} +  \theta_1x_1^{(i)} + \dots + \theta_jx_j^{(i)} + ... + \theta_nx_n^{(i)} - y^{(i)} \right] \\
	      &= \sum_{i=1}^m\left[ h_\theta(x^{(i)}) - y^{(i)} \right]x_j^{(i)}
	\end{aligned}\end{equation}
	所以,更新后的梯度下降公式为
	\begin{equation}\begin{aligned}
		\theta_j &:= \theta_j - \alpha \sum_{i=1}^m \left[ h_\theta(x^{(i)}) - y^{(i)} \right]x_j^{(i)}
	\end{aligned}\end{equation}
	其中,$\alpha$称为学习速率(learning rate),用于控制$\theta$前进的步伐,避免$\theta$走得太快(或太慢)\footnote{走得太慢需要的迭代很多次数才能收敛,走得太快有可能出现Cost Function值越来越大的情况,即Cost随着迭代次数增多不但不收敛反倒一直增大。\\
	在实践时,时常会画出Cost随迭代次数变化的图像(设置一定的次数间隔,不会每次都画),以实时观察学习效果}。\\
	梯度下降法每次计算都是找到当前所在位置中,下降最快的方向\footnote{至于为什么是当前所在位置中下降最快的方向就是梯度的定义了。不明白的请复习高数}。 \\
	如果计算进入了某个局部极小值,则很可能一直在这个局部极小值内出不来了\footnote{因为在这局部极小值中,各个方向的导数$\frac{\partial{J(\theta)}}{\theta_j}$都很小,其和也很小,故$\theta$一直徘徊在这局部极小值附近},除非你的步长比较大,帮助它跑出这个局部极小值\footnote{实际上,我们并不知道现在处在的是局部最小还是全局最小,为了避免Cost随迭代次数变大,我们也不会将步长设得太大},这就是梯度下降法的局限所在,它很容易陷入局部极小值中。 \\
	使用梯度下降法每次修正的是参数$\theta$,而不是输入的$x$或$y$\footnote{吴恩达在Coursera的课程中会画图帮助理解$\theta$的迭代过程,要注意,图像中的横轴是$\theta_j$,而不是$x_j$}。
\end{enumerate}


\subsubsection{随机梯度下降(SGD)}
由批梯度下降的式子可以发现,在每一次梯度下降的迭代过程中,我们都遍历了所有训练集,这将会耗费大量的性能,于是产生了随机梯度下降(SGD)方法,每次迭代只使用一个数据\footnote{因为每次只使用1个数据收敛得太慢了,因此一般不会使用此方法,大部分情况下会使用下面提到的迷你批梯度下降}。 \\
\begin{enumerate}
	\item Cost Function
	\begin{equation}\begin{aligned}
		J(\theta) = \frac{1}{2} \left[h_{\theta} {(x^{(i)})} - y^{(i)}\right]^2
	\end{aligned}\end{equation}

	\item 迭代方式
	\begin{equation}\begin{aligned}
	      \frac{\partial} {\partial \theta_j} J(\theta) &= \left[ h_\theta(x^{(i)}) - y^{(i)} \right]x_j^{(i)}
	\end{aligned}\end{equation}
	更新后
	\begin{equation}\begin{aligned}
		\theta_j &:= \theta_j - \alpha\left[ h_\theta(x^{(i)}) - y^{(i)} \right]x_j^{(i)}
	\end{aligned}\end{equation}
	如上所示,与批梯度下降不同,随机梯度下降每次迭代只用了一个数据(第$i$个数据),若$i$从1取到m,则完成了一次训练集的遍历\footnote{在训练中,为了使参数收敛,可以多次遍历训练集。比如我们的批梯度下降就每次都遍历了一遍训练集,随机梯度(或其他梯度下降方法)当然也可以多次遍历}。\\
	使用随机梯度下降虽然解决了批梯度下降耗费过多性能的问题,但是却带来了另一个问题:收敛太慢!由此,我们折中使用迷你批梯度下降(mini-batch GD)。\\
\end{enumerate}

\subsubsection{迷你批梯度下降(mini-batch GD)}
为了解决梯度下降每次都使用所有训练集导致的性能问题,以及随机梯度下降每次只使用一个数据导致的收敛太慢,我们可以只用mini-batch GD。每次只用一部分训练集进行训练。
\begin{enumerate}
	\item Cost Function \\
	略\footnote{与批梯度下降相比,只是求和的上限变了而已,从训练集大小$m$变成了一个比较合理的数字}

	\item 迭代方式 \\
	略
	\item 在很多第三方框架中使用的都是迷你梯度下降,故其在使用过程中都需要设置每次迭代使用的数据量大小。
\end{enumerate}

\subsubsection{其他}
\begin{enumerate}
	\item 在线性回归中,使用特征缩放可提高收敛速度
\end{enumerate}







			\subsection{线性代数知识}
\begin{enumerate}
	\item
	\begin{equation}
		\nabla_{\theta}J(\theta) = \left[\begin{matrix}
		\frac{\partial J}{\partial\theta_0} \\
		\frac{\partial J}{\partial\theta_1} \\
		\vdots \\
		\frac{\partial J}{\partial\theta_n} \\
		\end{matrix}\right], \quad \in {\rm I\!R}^{n+1}
	\end{equation}

	\item
	\begin{equation}
		\nabla_Af(A) = \left[ \begin{matrix}
			\frac{\partial f}{\partial A_{11}} & \frac{\partial f}{\partial A_{12}} & \dots & \frac{\partial f}{\partial A_{1n}} \\
			\frac{\partial f}{\partial A_{21}} & \frac{\partial f}{\partial A_{22}} & \dots & \frac{\partial f}{\partial A_{2n}} \\
			\vdots & \vdots & \ddots & \vdots \\
			\frac{\partial f}{\partial A_{n1}} & \frac{\partial f}{\partial A_{n2}}& \dots & \frac{\partial f}{\partial A_{nn}} \\
		\end{matrix}\right]
	\end{equation}

	\item 矩阵的迹的计算方式 \\
	如果矩阵$A$是方阵,则:
	\begin{equation}
		tr(A) = \sum_{i=1}^nA_{ii}
	\end{equation}

	\item 矩阵的迹的性质
	\begin{equation}
		tr(AB) = tr(BA)
	\end{equation}
	\begin{equation}
		tr(ABC) = tr(CAB) = tr(BCA)
	\end{equation}
	\begin{equation}
		tr(A^T) = tr(A)
	\end{equation}
	\begin{equation}
		tr(A+B) = trA + trB
	\end{equation}
	\begin{equation}
		traA = atrA
	\end{equation}

	\item 
	\begin{equation}
		\nabla_AtrAB = B^T
	\end{equation}
	\begin{equation}
		\nabla_{A^T}f(A) = (\nabla_Af(A))^T
	\end{equation}
	\begin{equation}
		\nabla_AtrABA^TC = CAB + C^TAB^T
	\end{equation}
	\begin{equation}
		\nabla_A|A| = |A|(A^{-1})^T
	\end{equation}
	\begin{equation}
		A^{-1} = \frac{(A^{'})^T}{|A|}
	\end{equation}
	其中,$A^{-1}$为矩阵的逆



\end{enumerate}



















			\subsection{梯度下降过程的矩阵表达}
\begin{enumerate}
	\item 
	\begin{equation}
		X = \left[\begin{matrix}
		-\!-  & (x^{(1)})^T & -\!- \\
		-\!- & (x^{(2)})^T & -\!- \\
		\vdots & \ddots & \vdots \\
		-\!- & (x^{(m)})^T & -\!- \\
		\end{matrix}\right] \quad \in {\rm I\!R}^{m \times (n+1)}
	\end{equation}

	\item 
	\begin{equation}
		\vec{y} = \left[\begin{matrix}
		y^{(1)} \\ y^{(2)} \\ \vdots \\ y^{(m)}
		\end{matrix}\right] \quad \in {\rm I\!R}^{m \times 1}
	\end{equation}

	\item 
	\begin{equation}
		X\theta - \vec{y} = \left[\begin{matrix}
		(x^{(1)})^T\theta \\ (x^{(2)})^T\theta \\ \vdots \\ (x^{(m)})^T\theta
		\end{matrix}\right] - \left[\begin{matrix}
		y^{(1)} \\ y^{(2)} \\ \vdots \\ y^{(m)}
		\end{matrix}\right] = \left[\begin{matrix}
		h_\theta(x^{(1)}) - y^{(1)} \\ h_\theta(x^{(2)}) - y^{(2)} \\ \vdots \\ h_\theta(x^{(m)}) - y^{(m)}
		\end{matrix}\right]  \quad \in {\rm I\!R}^{m \times 1}
	\end{equation}

	\item 
	\begin{equation}
		J(\theta) = \frac{1}{2}(X\theta - \vec{y})^T(X\theta - \vec{y})
	\end{equation}
	\footnote{这就是矩阵中平方的表达方式}

	\item 
	\begin{equation}
		\nabla_{\theta}J(\theta) = X^TX\theta - X^T \vec{y}
	\end{equation}
	为了得到最值,另上式值为0(极值点的方向倒数均为0),可得
	\begin{equation}
		\theta = (X^TX)^{-1}X^T\vec{y}
	\end{equation}






\end{enumerate}




























			\subsection{线性回归中使用最小二乘法的合理性解释}
\subsubsection{证明过程}
\begin{enumerate}
	\item 我们可以将实际的$y^{(i)}$分为两个部分:一部分被我们的计算模型所包括,为$\theta^Tx^{(i)}$;一部分未被我们的计算模型所包括,将其记为$\epsilon^{(i)}$,可得:
	\begin{equation}
		y^{(i)} = \theta^Tx^{(i)} + \epsilon^{(i)}
	\end{equation}

	\item 根据中心极限定理\footnote{中心极限定理大意:许多独立随机变量之和趋向于服从高斯分布。 更详细的待补充...},同时根据经验,我们可以假设$\epsilon^{(i)}$服从高斯分布$N(\mu, \sigma^2)$,故
	\begin{align}
		p(\epsilon^{(i)}) &= \frac{1}{\sqrt{2\pi}\sigma}e^{-\frac{\left(\epsilon^{(i)}-\mu\right)^2}{2\sigma^2}}  \\
		&\downarrow \\
		p(y^{(i)}|x^{(i)};\theta) &= \frac{1}{\sqrt{2\pi}\sigma}e^{-\frac{\left(y^{(i)}-\theta^Tx^{(i)}-\mu\right)^2}{2\sigma^2}}
	\end{align}
	\footnote{在数学上,对于连续型随机变量,$F(x)$表示随机变量的分布函数,$f(x)$表示其概率密度,二者表示的数学意义不一样,且$F'(x) = f(x)$;对于离散型来讲$P(X=x_i)=p_i$,两个值意义一样且都叫分布律}

	\item 写出其似然函数
	\begin{align}
		L(\theta) &= L(\theta; X, \vec{y}) = p(\vec{y}|X; \theta) \\
		&= \prod_{i=1}^{m}P(y^{(i)}|x^{(i)}; \theta) \\
		&= \prod_{i=1}^{m}\frac{1}{\sqrt{2\pi}\sigma}e^{-\frac{\left(y^{(i)}-\theta^Tx^{(i)}-\mu\right)^2}{2\sigma^2}} \\
		&= \left(\frac{1}{\sqrt{2\pi}\sigma}\right)^m\cdot e^{\sum_{i=1}^{m}{-\frac{\left(y^{(i)}-\theta^Tx^{(i)}-\mu\right)^2}{2\sigma^2}}}
	\end{align}
	\footnote{关于此式子的理解建议看看文档末尾附录中对似然函数的介绍。}

	\item 显然,求其对数后更好分析
	\begin{align}
		l(\theta) &= \ln{L(\theta)} \\
		&= \ln\left[\left(\frac{1}{\sqrt{2\pi}\sigma}\right)^m\cdot e^{\sum_{i=1}^{m}{-\frac{\left(y^{(i)}-\theta^Tx^{(i)}-\mu\right)^2}{2\sigma^2}}}\right] \\
		&= m\cdot\ln\frac{1}{\sqrt{2\pi}\sigma} - \sum_{i=1}^{m}\frac{\left(y^{(i)}-\theta^Tx^{(i)}-\mu\right)^2}{2\sigma^2} \\
		&= m\cdot\ln\frac{1}{\sqrt{2\pi}\sigma} - \frac{1}{{\sigma^2}}\cdot \frac{1}{2} \sum_{i=1}^{m}\left(y^{(i)}-\theta^Tx^{(i)}-\mu\right)^2
	\end{align}

	\item 由上式可知,若要最小化似然函数$l(\theta)$,等价于最小化$\frac{1}{2} \sum_{i=1}^{m}\left(y^{(i)}-\theta^Tx^{(i)}-\mu\right)^2$,此式子同样是最小二乘法的形式。令常数项$\mu = 0$便可得到前述线性回归Cost Function使用的最小二乘法,故在线性回归中使用最小二乘法计算$J(\theta)$是合理的

\end{enumerate}

\subsubsection{说明}
\begin{enumerate}
	\item 从上述结果中,我们可以发现,让$l(\theta)$取到最值时的$\theta$值与$\sigma^2$无关。后续说明指数分布族\&广义线性模型时将会用到此性质
\end{enumerate}















		\section{局部加权回归(LWR:Local Weight Regression)}
\subsection{概念}
\begin{enumerate}
	\item 局部加权回归思想由来 \\
	在普通的线性回归中,如果输入变量$X$与目标变量$y$之间并没有明显的线性关系,那么我们使用线性回归得到的拟合效果并不好。于是我们就想对我们的拟合方法做些改进,下面我们再复习下我们的线性回归:
	\begin{equation}
		h_{\theta}(x) = \sum_{j=0}^n \theta_jx_j = \theta_0x_0 + \theta_1x_1 + \theta_2x_2 + \dots +  + \theta_nx_n
	\end{equation}
	\begin{equation}\begin{aligned}
		J(\theta) = \frac{1}{2} \left[h_{\theta} {(x^{(i)})} - y^{(i)}\right]^2
	\end{aligned}\end{equation}
	\begin{equation}\begin{aligned}
		\theta_j &:= \theta_j - \alpha \sum_{i=1}^m \left[ h_\theta(x^{(i)}) - y^{(i)} \right]x_j^{(i)}
	\end{aligned}\end{equation}
	$h_\theta(x)$是假设函数,用来预测结果;$J(\theta)$为Cost Funtion,用来得到当前参数下预测值与实际值的差距,进而对$\theta$进行奖惩(通过迭代方式实施奖惩),从而得到拟合效果最好的$\theta$。\\
	若要得到不同的拟合效果,一方面我们可以更改$h_\theta(x)$,使用不同的拟合函数;另一方面,我们可以保持原来的拟合函数,通过改变迭代方式实施不同的奖惩措施,最终得到不同的$\theta$;又或者,我们仅改变Cost Function,给不同位置的点$(x^{(i)}, y^{(i)})$不同的权重。\\
	若要更改$h_\theta(x)$,我们可以添加$x^2, x^3, \dots$或使用其他的拟合函数,这在其他章节会讲到了,不在我们本次的讨论内容中(更改迭代方式的方法也会在其他章节讲到);下面我们讲下如何更改$J(\theta)$来获取不同的拟合效果:\\
	% 有上式可知,若要改善拟合效果,一方面可以从$x$入手,如添加更高阶的拟合方式,由此来得到更准确的$\theta$来拟合;\\
	我们可以发现,上述线性回归根本思想其实就是给每个特征赋予不同的权值($x_j$就是特征,$\theta_j$就是权值)。其赋权只在$h_\theta(x)$中,在$J(\theta)$中并没有赋权,且只考虑到了给每个特征$\theta_j$赋予不同的权值,现在,我们考虑给$J(\theta)$赋权,且让这个权值与该数据点所处位置相关,这就是局部加权回归(若采用其他的赋权方法也可以,但这就不叫局部加权回归了)。\\
	下面,我们讲讲局部加权回归的计算方法
\end{enumerate}

\subsection{计算方法}
\begin{enumerate}
	\item 设计权值函数
	\begin{equation}
		\omega^{(i)} = e^{-\frac{(x^{(i)}-x)^2}{2}}
	\end{equation}
	$\omega^{(i)}$就是我们为局部加权回归量身定做的加权函数. \\
	其具有正态分布函数的形式,但没有其代表的意义(当然,我们也可以使用其他函数作为权值函数,但这就不叫局部加权回归了)。\\
	其中,$x$就是我们要预测的点,$\omega^{(i)}$在$|x^{(i)}-x|$较小时,其值约等于1;在$|x^{(i)}-x|$较大时,其值约等于0。\\
	\item Cost Function
	\begin{equation}
		J(\theta) = \sum_{i}\omega^{(i)}(y^{(i)} - \Theta^Tx^{(i)})^2
	\end{equation}
	之后,通过与一般线性回归一样的方式不断地拟合,让$J(\theta)$的值最小便可得到局部加权回归的结果。最终的效果中,距离要预估的点较近的点对结果影响较大,距离要预估的点较远的点对结果影响较小\footnote{虽然$h_\theta(x)$一样,但是因为计算误差的方式变化导致惩罚机制变化,以此来改变不同位置点的影响}。
	\item 优化 \\
	添加带宽参数(Bandwidth Parameter)$\tau^2$,来控制不同距离对预估结果的影响
	\begin{equation}
		\omega^{(i)} = e^{-\frac{(x^{(i)}-x)^2}{2\tau^2}}
	\end{equation}
	
\end{enumerate}

\subsection{其他注意事项}
\begin{enumerate}
	\item 使用局部加权回归得到的仍旧是一条直线$h_{\theta}(x)= \sum_{j=0}^n\theta_jx_j^{(i)}$
	\item 局部加权回归是一个非参数学习算法\footnote{参数学习算法: 对于线性回归算法,一旦拟合出适合训练数据的参数$\theta$,保存这些参数$\theta$,对于之后的预测,不需要再使用原始训练数据集; 非参数学习算法: 对于局部加权线性回归算法,每次进行预测都需要全部的训练数据(每次进行的预测得到不同的参数$\theta$),没有固定的参数$\theta$}。
	\item 局部加权回归同样会有过拟合、欠拟合的问题。
	\item 因为局部加权函数在对$\theta$进行奖惩时,用到了我们要预测的那个点,最终的$\theta$受到该点的影响,所以,如果我们要预测其他点,需要重新进行一次局部加权回归。这将导致此算法不适合所要预测的点一直变化的情况。
\end{enumerate}









		\section{逻辑回归}


			\subsection{感知器算法}
\begin{enumerate}
	\item 将逻辑回归中的$g(z)$换成
	\[ g(z)=\begin{cases}
	1 \quad z \geq 0, \\
	0 \quad z <0
	\end{cases} \]
	即$h_\theta(x) = g(\theta^Tx)$。其得到的$h_\theta(x)$值只有$0$或$1$。 \\
	其他均与逻辑回归一致,这就变成了感知器算法
	\item 在逻辑回归迭代方式的证明中,用到了sigmoid Function $g^{'}{(x)} = g(z)\left[1-g(z)\right]$的性质,对感知器算法中的$g(z)$,该式子同样成立。经证明后可发现,感知器算法的迭代方式仍旧与逻辑回归一样
	\begin{align}
		\theta_j :=  \theta_j - \alpha \sum_{i=1}^{m} \left[h_\theta(x^{(i)}) - y^{(i)} \right]x_j^{(i)}
	\end{align}
\end{enumerate}

















			\subsection{牛顿方法}

\subsubsection{牛顿方法的思路}
\begin{enumerate}
	\item 先介绍如何使用牛顿方法得到某函数的0点,因为在极值出现的地方其导数必定为0,因此,可以用牛顿方法找到导数为0的点,这个点就有可能是个极值点。(这就是不论我们要找的是最大值还是最小值,最终的式子都是一样的原因。)
	\item 注意事项
	\begin{enumerate}
		\item 后续再介绍牛顿方法时,仍旧使用逻辑回归作为例子,所以,其似然函数$l(\theta)$还是与逻辑回归一样
	\end{enumerate}
\end{enumerate}

\subsubsection{牛顿方法讲解}
\begin{enumerate}
	\item 牛顿方法介绍: 如下面示意图所示,我们要求的是曲线$y=f(x)$与直线$y=y_{final}$的交点$(x_{final}, y_{final})$的值,为此,我们先任取曲线上的一点$(x_{ori}, y_{ori})$,求得在该点的切线,记为$y=f'(x_{ori})x + b$,然后得到该切线与$y=y_{final}$的交点$(x_{new}, y_{final})$,得到$x_{new}$后,我们又可以得到曲线上新的一点$(x_{new}, f(x_{new}))$;用新的点$(x_{new}, f(x_{new}))$代替之前的$(x_{ori}, y_{ori})$,重复之前的步骤。就这样,一次一次地重复后,我们就能一直逼近我们所要的$(x_{final}, y_{final})$了。
	\begin{figure}[htbp]
		\centering
		\includegraphics[scale=0.9]{images/牛顿方法示例图片}
		\caption{牛顿方法示例图}
	\end{figure}

	\item 在逻辑回归中使用牛顿方法的过程
	\begin{enumerate}
		\item 由上图可知,$\tan\alpha = \frac{y_{ori}-y_{final}}{x_{ori}-x_{new}} = \frac{f(x_{ori})-f(x_{final})}{x_{ori}-x_{new}} = f'(x_{ori})$,可以得到
		\begin{equation}
			x_{new} := x_{ori} - \frac{f(x_{ori})-f(x_{final})}{f'(x_{ori})}
		\end{equation}
		
		\item 此时,我们得到了一个新的点$(x_{new}, y_{new})$,相比与原来的点$(x_{ori},y_{ori})$,这个点离最终的$(x_{final}, y_{final})$更近了。

		\item 特殊地,在逻辑回归中,我们要得到的是似然函数$l(\theta)$的极值,极值点的导数为$0$。于是,为了在逻辑回归中使用牛顿方法,我们用$l'(\theta)$代替$f(x)$,且$l'(\theta_c) = 0$,于是,上式变成
		\begin{align}
			\theta_{new} &:= \theta_{ori} - \frac{l'(\theta_{ori})-l'(\theta_{final})}{l''(\theta_{ori})} \\
			&:=  \theta_{ori} - \frac{l'(\theta_{ori})}{l''(\theta_{ori})}
		\end{align}
		这就是逻辑回归使用牛顿方法时的迭代方式
	\end{enumerate}

	\item 矩阵表达形式 \\
	将上面的式子改写为矩阵表达的形式,如下
	\begin{equation}
		\theta := \theta - H^{-1}\nabla_{\theta}l(\theta)
	\end{equation}
	其中,$H$称为Hessian,是一个$n*n$矩阵
	\begin{equation}
		H_{ij} = \frac{\partial^2l(\theta)}{\partial\theta_i\partial\theta_j}
	\end{equation}

	\item 关于牛顿方法的思考
	\begin{enumerate}
		\item 因为不论我们要得到的是最大值还是最小值,其极值点的导数均为$0$,因此,虽然上述逻辑回归中,我们要得到的时似然函数的极大值,但是,若在其他问题中我们要的得到的时极小值,最终的迭代方式还是一样的。
		\item \item 但是,通过此方法找到的只是极值点(也可能连极值点都不是),如何保证其为全局最优(最大或最小)呢?。{\color{red}{此项待研究。。}}
	\end{enumerate}

	\item 牛顿方法与梯度下降方法的异同
	\begin{enumerate}
		\item 牛顿方法可以在很少的迭代次数就收敛,而梯度下降要收敛需要迭代的次数可能较多
		\item 但是,牛顿方法每次迭代的计算量较大(因为要计算$H^{-1}$),这在特征维度$n$较大时将会耗费较多的性能
		\item 综上,在特征维度$n$不是太大时,使用牛顿方法能够很快就收敛;但是若特征维度$n$很大时,虽然使用牛顿方法需要迭代的次数较少,但是它每次耗费的计算量太大,整个学习的时间不一定能够比梯度下降少。
	\end{enumerate}
\end{enumerate}















			
		\section{广义线性模型}
\subsection{指数分布族}
\begin{enumerate}
	\item 指数分布族的一般形式
	\begin{equation}
		p(y;\eta) = b(y)e^{\eta^T T(y) - a(\eta)}
	\end{equation}
	其中,$\eta$称为natural parameter(自然参数???); \\
	$T(y)$称为sufficient statistic(充分统计量??),大部分情况下,$T(y) = y$,在$T(y)=y$时,$\eta$往往为标量; \\
	$a(\eta)$称为 log partition function;
	\item 
\end{enumerate}

\subsection{伯努利分布于高斯分布中,GLM各部分的值}
主要是将伯努利分布(即$0-1$分布)与高斯分布的式子尽可能地转化成指数分布族的一般形式,然后在根据该一般形式写出$\eta$,$T(y)$,$a(\eta)$的形式。\\
\begin{enumerate}
	\item 伯努利分布
	\begin{align}
		P(y;p) &= p^y(1-p)^{1-y} \\
		&= e^{y\ln p + (1-y)\ln(1-p)} \\
		&= e^{ln{\frac{p}{1-p}}y + \ln(1-p)}
	\end{align}
	与广义线性模型的一般形式比较,可以发现:
	\begin{align}
		b(y) &= 1 \\
		\eta^T &= \ln{\frac{p}{1-p}} = \eta \rightarrow p = \frac{e^{\eta}}{1+e^{\eta}} \\
		T(y) &= y \\
		a(\eta) &= -\ln(1-p) = -\ln{(1-\frac{e^{\eta}}{1+e^{\eta}})} = -\ln{\frac{1}{1+e^\eta}} \\
		&= \ln(1+e^\eta)
	\end{align}

	\item 高斯分布 \\
	在证明使用最小二乘法作为线性回归的Cost Fuction\footnote{最大化$\theta$等同于最大化$\frac{1}{2} \sum_{i=1}^{m}\left(y^{(i)}-\theta^Tx^{(i)}-\mu\right)^2$}的可行性时提到: $J(\theta)$取到最值时的$\theta$值与$\sigma^2$无关(但与$\mu$有关),故在下面的证明中,我们另$\sigma^2=1$,以简化计算
	\begin{align}
		P(y;\mu) &= \frac{1}{\sqrt{2\pi}}e^{-\frac{1}{2}\left(y-\mu\right)^2} \\
		&= \frac{1}{\sqrt{2\pi}} e^{-\frac{y^2-2\mu y +\mu^2}{2}} \\
		&= \frac{1}{\sqrt{2\pi}}e^{-\frac{y^2}{2}} e^{\mu y-\frac{\mu^2}{2}}
	\end{align}
	故:
	\begin{align}
		b(y) &= \frac{1}{\sqrt{2\pi}}e^{-\frac{y^2}{2}} \\
		\eta^T &= \mu = \eta \\
		T(y) &= y \\
		a(\eta) &= -1\cdot (-\frac{\mu^2}{2}) = \frac{\mu^2}{2} = \frac{\eta^2}{2}
	\end{align}

	\item 注意事项
	\begin{enumerate}
		\item $\eta^T$要在$\eta$为标量时才能直接等于$\eta$,故上面的式子中是在得到$\eta^T$为标量后再写$=\eta$
		\item $a(\eta)$通常不会直接得到关于$\eta$的函数,需要通过$\eta^T$来与$\eta$关联,得到$\eta$的函数
	\end{enumerate}
\end{enumerate}

\subsection{如何构造一般线性模型}






% 





















			\subsection{Softmax Regression}
\subsubsection{关于$\phi$的说明}
\begin{enumerate}
	\item 对于总共有$k$类的分类问题,因为每个分类出现的概率和为1,故仅有$k-1$维特征是相互独立的
	\item 我们将$p(y=i;\phi)$记为$\phi_i$;于是$\phi_k=p(y=k;\phi)=1-\sum_{i=1}^{k-1}\phi_i$。
\end{enumerate}

\subsubsection{关于$T(y)$的表示法说明}
\begin{enumerate}
	\item 为了描述$T(y)$属于多分类问题中的那一类,我们用$k-1$为向量来描述$T(y)$,其中,若$T(y)$属于第$i$类,则第$i$维值为1,其余均为0。示例如下:
	\begin{equation}
		T(2) = \left[\begin{matrix} 0 \\ 1 \\ 0 \\ \vdots \\ 0 \end{matrix}\right], \quad
		T(1) = \left[\begin{matrix} 1 \\ 0 \\ 0 \\ \vdots \\ 0 \end{matrix}\right], \quad 
		T(k-1) = \left[\begin{matrix} 0 \\ 0 \\ 0 \\ \vdots \\ 1 \end{matrix}\right]
	\end{equation}
	于是,$T(k)$就是个$\vec{0}$向量,
	\begin{equation}
		T(k) = \left[\begin{matrix} 0 \\ 0 \\ 0 \\ \vdots \\ 0 \end{matrix}\right]
	\end{equation}
	\item 为了描述方便,我们使用以下新的表示方法
	\begin{enumerate}
		\item $1\{\cdot\}$,在$\cdot$的值为$True$时,$1\{True\}=1$,否则$1\{False\}=0$。如$1\{2=3\}=0$,$1\{2=2\}=1$
		\item 在以上表示法下,$T(y)$可表示成以下形式
		\begin{align}
			\left(T(y)\right)_i &= 1\{y=i\} \\
			& \downarrow \\
			E\left[\left(T(y)\right)_i\right] &= P(y=i)=\phi_i
		\end{align}
	\end{enumerate}
\end{enumerate}

\subsubsection{证明多项式分布属于指数族分布}
\begin{enumerate}
	\item 在多分类问题中,$p(y;\phi)$表示的含义,及其表示方式
	\begin{enumerate}
		\item 与逻辑回归时的二分类一样,$p(y=k;\phi)$表示的是取到第$k$类的概率,用$p(y;\phi)$将$p(y=1;\phi), p(y=2;\phi), \dots, p(y=k;\phi)$表示成一个表达式
		\item 参考逻辑回归的方式,我们将$p(y;\phi)$表示为
		\begin{align}
			p(y;\phi) &= \phi_{1}^{1\{y=1\}}\phi_{2}^{1\{y=2\}}\dots\phi_{k}^{1\{y=k\}} \\
			&= \phi_{1}^{1\{y=1\}}\phi_{2}^{1\{y=2\}}\dots\phi_{k}^{1-\sum_{i=1}^{k-1}1\{y=i\}} \\
			&= \phi_{1}^{\left(T(y)\right)_1}\phi_{2}^{\left(T(y)\right)_2}\dots\phi_{k}^{1-\sum_{i=1}^{k-1}\left(T(y)\right)_i} \\
			&= e^{\left(T(y)\right)_1\ln\phi_1 + \left(T(y)\right)_2\ln\phi_2 + \dots + \left(T(y)\right)_{k-1}\ln\phi_{k-1} + \left[1-\sum_{i=1}^{k-1}\left(T(y)\right)_i\right]\ln\phi_k } \\
			&= e^{\left(T(y)\right)_1\ln\frac{\phi_1}{\phi_k} + \left(T(y)\right)_2\ln\frac{\phi_2}{\phi_k} + \dots + \left(T(y)\right)_{k-1}\ln\frac{\phi_{k-1}}{\phi_k} + \ln\phi_k} \\
			&= b(y)e^{\eta^T T(y) - a(\eta)}
		\end{align}
		\end{enumerate}
	\item 从上式可得
	\begin{align}
		b(y) &= 1 \\
		\eta^TT(y) &= \left(T(y)\right)_1\ln\frac{\phi_1}{\phi_k} + \left(T(y)\right)_2\ln\frac{\phi_2}{\phi_k} + \dots + \left(T(y)\right)_{k-1}\ln\frac{\phi_{k-1}}{\phi_k} \\
		&\downarrow \\
		\eta^T &= \left[\begin{matrix}
		\ln\frac{\phi_1}{\phi_k} \\ \ln\frac{\phi_2}{\phi_k} \\ \vdots \\ \ln\frac{\phi_{k-1}}{\phi_k}
		\end{matrix}\right] \\
		a(\eta) &= -\ln\phi_k
	\end{align}
\end{enumerate}


\subsubsection{使用Softmax进行分类}
\begin{enumerate}
	\item 对于$\eta$中的某一项
	\begin{align}
		\eta_i &= \ln{\frac{\phi_i}{\phi_k}} \\
		&\downarrow \\
		e^{\eta_i} &= \phi_i \\
		&\downarrow \\
		\phi_k e^{\eta_i} &= \phi_i \\
		\phi_k\sum_{i=1}^{k}e^{\eta_i} &= \sum_{i=1}^{k}\phi_i = 1 \\
		&\Downarrow \\
		\phi_i &= \phi_k\cdot e^{\eta_i} \\
		&= e^{\eta_i} \cdot \frac{1}{\sum_{j=1}^{k}e^{\eta_j}}
	\end{align}
	如上所示,$\phi_i = \frac{e^{\eta_i}}{\sum_{j=1}^{k}e^{\eta_j}}$称为Softmax函数
	\item 最后,再使用前面的假设三:$\eta_i = \theta_i^Tx, \quad i=1,2, \dots, k-1, \quad \theta_i \in {\rm I\!R}^{n+1}$,可得到
	\begin{align}
		p(y=i|x; \theta) &= \phi_i \\
		&= \frac{e^{\eta_i}}{\sum_{j=1}^{k}e^{\eta_j}} \\ 
		&= \frac{e^{\theta_i^Tx}}{\sum_{j=1}^{k}e^{\theta_j^Tx}}
	\end{align}
	\item 故
	\begin{align}
		h_\theta(x) &= E\left[T(y)|x;\theta\right] \\
		&= E\left[\left[\begin{matrix} 1\{y=1\} \\ 1\{y=2\} \\ \vdots 1\{y=k-1\} \end{matrix}\right|x;\theta\right] \\
		&= \left[\begin{matrix}\phi_1 \\ \phi_2 \\ \vdots \\ \phi_{k-1} \end{matrix}\right] \\ 
		&= \left[\begin{matrix}
		\frac{e^{\theta_1^Tx}}{\sum_{j=1}^{k}e^{\theta_j^Tx}} \\
		\frac{e^{\theta_2^Tx}}{\sum_{j=1}^{k}e^{\theta_j^Tx}} \\
		\vdots \\
		\frac{e^{\theta_{k-1}^Tx}}{\sum_{j=1}^{k}e^{\theta_j^Tx}}
		\end{matrix}\right]
	\end{align}
	如上,最终$h_\theta(x)$可得到$k-1$维向量,通过总概率为1得到第$k$为的概率值,由此可以得到没一种类别的概率值
	\item 以上得到多分类中每种分类概率值的方法就叫做Softmax回归(Softmax Regression)
\end{enumerate}


\subsubsection{参数$\theta$的拟合方式}
使用对数似然估计方法
\begin{align}
	l(\theta) &= \sum_{i=1}^{m}\ln p\left[y^{(i)}|x^{(i)};\theta\right] \\
	&= \sum_{i=1}^{m} \ln \prod_{l=1}^{k}\left[
	\frac{e^{\theta_l^Tx^{(i)}}}{\sum_{j=1}^{k}e^{\theta_l^Tx^{(i)}}}
	\right]^{1\{y^{(i)}=l\}}
\end{align}
{\color{red}{需后续待补充}}













		\section{生成学习算法}
\subsection{生成学习算法简介}

\subsubsection{判别学习算法\&生成学习算法}
\begin{enumerate}
	\item 判别学习算法:计算$p(y|x)$,例如逻辑回归中计算$p(y=1|x)$与$p(y=0|x)$
	\item 生成学习算法:计算$p(x|y)$与$p(y)$,然后通过贝叶斯公式$p(y|x) = \frac{p(x|y)p(y)}{p(x)}$计算得到$p(y|x)$
	\begin{enumerate}
		\item 式子中的$p(y)$就是每个$y=k$所占的比例,既然我们有了数据,就可以用其频率当作其比例
		\item 不论$p(y|x)$中$y$的值为多少,式中的$p(x)$均是一样的值(可通过全概率公式计算得到),因此,对于$p(y|x)$较大的$y=k$,$p(x|y)p(y)$也较大,因此,在最优化过程中,可以不计算$p(x)$,于是优化过程变为
		\begin{align}
			arg\max{p(y|x)} &= arg \max_y{\frac{p(x|y)p(y)}{p(x)}} \\
			&= arg\max_y{p(x|y)p(y)}
		\end{align}
		省去了计算$p(x)$的过程
	\end{enumerate}
	
	\item 判别学习算法是给如一系列特征,得到这些特征应该是哪一种结果;生成学习算法学习的是这一种结果应该有什么特征

\end{enumerate}




















			\subsection{高斯判别分析}
以下内容中均假设$p(x|y)$服从多重正态分布(高斯分布)

\subsubsection{多重正态分布简介}
\begin{enumerate}
	\item $n$重正态分布可由均值向量$\mu \in {\rm I\!R}^n$和协方差矩阵$\Sigma \in {\rm I\!R}^{n\times n}$确定,记做$N(\mu, \Sigma)$,表示为:
	\begin{equation}
		p(x; \mu, \Sigma) = \frac{1}{(2\pi)^{\frac{n}{2}}|\Sigma|^\frac{1}{2}}e^{-\frac{1}{2}(x-\mu)^T\Sigma^{-1}(x-\mu)}
	\end{equation}
	其中,$|\Sigma|$是协方差矩阵$\Sigma$的行列式;$\Sigma=\mathrm{Cov}(X) = E(XX^T) - E(X)\left[E(X)\right]^T$,$X$的期望为$E(X) = \int_{x}xp(x;\mu,\Sigma)dx = \mu$
\end{enumerate}

\subsubsection{高斯判别分析模型}
\begin{enumerate}
	\item 高斯判别分析模型针对的是连续性随机变量,如果是离散型随机变量请使用后续会讲到的朴素贝叶斯
	\item 我们假设$p(x|y)$服从多远高斯分布,于是
	\begin{align}
		y & \sim B(1, \phi) = Bernoulli(\phi) \\
		x|y=0 &\sim N(\mu_0, \Sigma) \\
		x|y=1 &\sim N(\mu_1, \Sigma)
	\end{align}
	于是,其概率密度为:
	\begin{align}
		p(y) &= \phi^y(1-\phi)^{1-y} \\
		p(x|y=0) &= \frac{1}{(2\pi)^{\frac{n}{2}}|\Sigma|^\frac{1}{2}}e^{-\frac{1}{2}(x-\mu_0)^T\Sigma^{-1}(x-\mu_0)} \\
		p(x|y=1) &= \frac{1}{(2\pi)^{\frac{n}{2}}|\Sigma|^\frac{1}{2}}e^{-\frac{1}{2}(x-\mu_1)^T\Sigma^{-1}(x-\mu_1)}
	\end{align}
	在上式中,$\phi, \mu_0, \mu_1, \Sigma$就是我们要求的参数(注意,有2个$\mu$,共享1个$\Sigma$),其对数似然函数为:
	\begin{align}
		l(\phi, \mu_0, \mu_1, \Sigma) &= \log \prod_{i=1}^{m} p\left(x^{(i)}, y^{(i)}; \phi, \mu_0, \mu_1, \Sigma\right) \\
		&= \log \prod_{i=1}^{m} p\left(x^{(i)}|y^{(i)}; \mu_0, \mu_1, \Sigma\right)p\left(y^{(i)};\phi\right)
	\end{align}
	注意$p(x|y)$中并没有参数$\phi$,$p(y)$中只有参数$\phi$
	\item 求解似然函数后可得参数如下
	\begin{align}
		\phi &= \frac{1}{m} \sum_{i=1}^{m}1\{y^{(i)=1}\} \\
		\mu_0 &= \frac{\sum_{i=1}^{m}1\{y^{(i)}=0\}x^{(i)}}{\sum_{i=1}^{m}1\{y^{(i)}=0\}} \\
		\mu_1 &= \frac{\sum_{i=1}^{m}1\{y^{(i)}=1\}x^{(i)}}{\sum_{i=1}^{m}1\{y^{(i)}=1\}} \\
		\Sigma &= \frac{1}{m}\sum_{i=1}^{m}\left(x^{(i)}-\mu_{y^{(i)}}\right)\left(x^{(i)}-\mu_{y^{(i)}}\right)^T
	\end{align}
	下面解释下以上各参数表达式的意义: \\
	$\phi$:表示所有数据中$y=1$所占的比例; \\
	$\mu_0$: 其分母表示所有数据中$y=0$的数目,分子表示$y=0$对应的$x$的和;$\mu_1$同理 \\
	$\Sigma$: 协方差矩阵,前面已说过,不再赘述
	\item 说明
	\begin{enumerate}
		\item 使用通过高速判别分析得到的模型得到的是两个高斯分布
		\item 这两个高速分布有共同的协方差$\Sigma$,但是其期望$\mu$不一样,若画出其等高线,在图形上表现为两个等高线形状(由$\Sigma$决定)一样,但是其中心(由$\mu$决定)不一样。
	\end{enumerate}
\end{enumerate}

\subsubsection{高斯判别分析\&逻辑回归}
\begin{enumerate}
	\item 如果我们将$\phi, \mu_0, \mu_1, \Sigma$看出$x$的函数,进行整理后最终可得到
	\begin{equation}
		p(y=1|x; \phi, \mu_0, \mu_1, \Sigma) = \frac{1}{1+e^{-\theta^Tx}}
	\end{equation}
	其中,$\theta$是$\phi, \mu_0, \mu_1, \Sigma$的函数
	\item 在我们的证明过程中,我们假设了$p(x|y)$服从高斯分布,相较于逻辑回归,GDA做了更强的假设。
	\item 如果$p(x|y)$服从多重高斯分布,则$p(y|x)$服从逻辑回归;但反,从$p(y|x)$服从逻辑回归无法得到$p(x|y)$服从多重高斯分布。这说明GDA需要更多的条件,其做了更强的假设。
	\item 在两种算法的准确性上:如果$p(x|y)$确实服从或近似服从高斯分布(即我们对其做的假设正确),那么GDA能够更好地进行预测;但是,若我们的假设错误(即$p(y|y)$实际并不服从高斯分布,可能是泊松分布或其他),此时,逻辑回归能得到更好的效果
	\item 在数据集$m$很大时,GDA可以得到比大部分算法更好的结果(包括逻辑回归);甚至在数据集较小时,GDA也往往比逻辑回归好
	\item 通常来讲,使用生成学习算法需要的数据更少(因为做了高斯分布的假设,从高斯分布继承了很多性质),因为大部分情况下数据虽然不是精确的高斯分布,但近似服从高斯分布。
	\item 因为逻辑回归的假设更弱(可理解为需要的条件更少),因此其具有更好的鲁棒性(可理解为适应能力更好),即使$p(y|x)$并不符合也不近似符合高斯分布,逻辑回归也能较好地预测。
	\item 使用逻辑回归可能需要比较多的数据样本
\end{enumerate}
















			\subsection{朴素贝叶斯}
\begin{enumerate}
	\item 在GDA中,随机变量$X$是连续的,当随机变量是离散值时,我们使用朴素贝叶斯来进行预测
	\item 朴素贝叶斯举例介绍:以通过邮箱中的文本预测是否为垃圾邮件为例,$y=1$表示该邮件是垃圾邮件
	\begin{enumerate}
		\item 对每封邮件建立词向量\footnote{此处的词向量比较简单,与文本分析(NLP)中的词向量不一样,也不是One-Hot的形式},每个词在向量中占据一个位置,若该词存在,则将该位的值设为1,若不存在则设为0,得到如下向量
		\begin{align}
			x = \left[\begin{matrix}1 \\ 0 \\ 0 \\ \vdots \\ 1 \\ \vdots \\ 0 \end{matrix}\right] \quad
			\begin{matrix}a \\ ad \\ address \\ \vdots \\ buy \\ \vdots \\ zygmurgy \end{matrix}
		\end{align}
		如上所示,该邮件中有a, buy等,没有ad, address, zygmurgy等。
		\footnote{一般来说,建议通过扫描训练集来获取可能出现的词(而不是从字典中获取),这样做一方面可以减小建模时的特征数,节省计算量和存储空间;另一方面还可以避免出现一些字典中没有的词(如CS299,或一些人名等)\\
		另外,还可以将一些无实意、出现频率又比较高的词去除,如a, an, the等}
		\item 若要使用多项式分布来进行预测,按照之前的逻辑回归方式,在词向量维度较多时并不可取(PS:讲义中的$2^{50000}-1$看起来似乎有问题,对$50000$个词建模只需要有$50000+1$个参数,哪来的$2^{50000}-1$??{\color{red}{待研究...}})
		\item 为了更方便地进行建模,我们做了一个假设:假设对于给定的$y$,$x_i$($x_i$指词向量中的第$i$个词)是相互条件独立的,即$p(x_2|y, x_1) = p(x_2|y)$,给定的$x_1$条件并不影响$p(x_2|y)$。此假设称为{\color{blue}{朴素贝叶斯假设}}。由此得到的分类器称为{\color{blue}{朴素贝叶斯分类器}}。
		\item 根据以上假设,我们可以得到以下式子:
		\begin{align}
			p(x_1, \dots, x_n|y) &= p(x_1|y)p(x_2|y, x_1)p(x_3|y, x_1, x_2)\dots p(x_n|y, x_1, x_2, \dots, x_{n-1}) \\
			&= p(x_1|y)p(x_2|y)p(x_3|y)\dots p(x_n|y) \\
			&= \prod_{i=1}^{n}p(x_i|y)
		\end{align}
		\item 与GDA类似,上式由参数$\phi_{i|y=1}=p(x_i=1|y=1)$,$\phi_{i|y=0}=p(x_i=1|y=0)$,$\phi_y=p(y=1)$决定。似然函数可表示为:
		\begin{align}
			L(\phi_y, \phi_{i|y=0}, \phi_{i|y=1}) = \prod_{i=1}^{n}p(x^{(i)}|y^{(i)})
		\end{align}
		\item 通过最大似然估计,可得
		\begin{align}
			\phi_{j|y=1} &= \frac{\sum_{i=1}^{m}1\{x_j^{(i)}=1 \cap y^{(i)}=1\}}{\sum_{i=1}^{m}1\{y^{(i)}=1\}} \\
			\phi_{j|y=0} &= \frac{\sum_{i=1}^{m}1\{x_j^{(i)}=1 \cap y^{(i)}=0\}}{\sum_{i=1}^{m}1\{y^{(i)}=0\}} \\
			\phi_y &= \frac{\sum_{i=1}^{m}1\{y^{(i)}=1\}}{m}
		\end{align}
		其中: \\
		$\cap$表并集,即两个条件同时满足 \\
		$\phi_{j|y=1}$表示出现词$x_j$的垃圾邮件在所有垃圾邮件中所占的比例 \\
		$\phi_{j|y=0}$表示出现词$x_j$的非垃圾邮件在所有非垃圾邮件中所占的比例 \\
		$\phi_y$表示垃圾邮件占所有邮件的比例
		\item 使用贝叶斯公式计算$p(y|x)$
		\begin{align}
			p(y=1|x) &= \frac{p(x|y=1)p(y=1)}{p(x)} \\
			&= \frac{\prod_{i=1}^{n}p(x_i|y=1)p(y=1)}{\left[\prod_{i=1}^{n}p(x_i|y=1)\right]p(y=1)+\left[\prod_{i=1}^{n}p(x_i|y=0)\right]p(y=0)}
		\end{align}
		对$p(y=0|x)$同理,通过比较那个概率更高\footnote{所以,其实$p(x)$并没有必要求,虽然也不难求}来判断是否为垃圾邮件。
		\item 虽然以上过程中只是二分类,但实际上,对于多分类也是以上的套路;
		\item 对于连续型随机变量,也可以对其分段离散化,然后使用朴素贝叶斯进行预测
	\end{enumerate}
\end{enumerate}

\subsection{拉普拉斯平滑}
\subsubsection{背景介绍}
\begin{enumerate}
	\item 在朴素贝叶斯中,若在预测时出现了一个我们训练集中没有的词,那么预测结果为
	\begin{align}
		\phi_{{x_j}|y=1} &= \frac{\sum_{i=1}^{m}1\{x_j^{(i)}=1 \cap y^{(i)}=1\}}{\sum_{i=1}^{m}1\{y^{(i)}=1\}} \\
		&= 0 \\
		\phi_{{x_j}|y=1} &= \frac{\sum_{i=1}^{m}1\{x_j^{(i)}=1 \cap y^{(i)}=0\}}{\sum_{i=1}^{m}1\{y^{(i)}=0\}} \\
		&= 0 \\
		p(y=1|x) &= \frac{\prod_{i=1}^{n}p(x_i|y=1)p(y=1)}{\left[\prod_{i=1}^{n}p(x_i|y=1)\right]p(y=1)+\left[\prod_{i=1}^{n}p(x_i|y=0)\right]p(y=0)} \\
		&= \frac{0}{0}
	\end{align}
	此时出现了$\frac{0}{0}$的情况,没法进行预测。为了解决此问题,我们可以使用拉普拉斯平滑。\footnote{吴恩达的视频中,举了一个已知前面5次比赛都失败,预测第6次失败的概率,可以看下。}
\end{enumerate}

\subsubsection{拉普拉斯平滑介绍}
\begin{enumerate}
	\item $\frac{0}{0}$出现的原因 \\
	由前面$\phi_j = \frac{\sum_{i=1}^{m}1\{z^{(i)}=j\}}{m}$可知,$\phi_j$为出现$z^{(i)}=j$的数据占总数据的比例,在$z^{(i)}=j$在训练集中未出现过时,就会出现$\phi_j=0$\footnote{在$=0$处频数为0,在$=1$处频数也为0},从而导致出现$\frac{0}{0}$的情况,为了解决此问题,我们队$\phi_j$稍作更改:
	\begin{equation}
		\phi_j = \frac{\sum_{i=1}^{m}1\{z^{(i)}=j\}+1}{m+k}
	\end{equation}
	经过此方式处理后:
	\begin{enumerate}
		\item $\sum_{j=1}^{k}\phi_j = 1$仍旧成立
		\item 对任意$\phi_j \neq 0$
	\end{enumerate}
\end{enumerate}














			\subsection{文本分类}
\begin{align}
	\phi_{k|y=1} &= \frac{\sum_{i=1}^{m}\sum_{j=1}^{n}1\{x_j^{(i)}=k \cap y^{(i)}=1\}+1}{\sum_{i=1}^{m}1\{y^{(i)}=1\}n_i + |V|} \\
	\phi_{k|y=0} &= \frac{\sum_{i=1}^{m}\sum_{j=1}^{n}1\{x_j^{(i)}=k \cap y^{(i)}=0\}+1}{\sum_{i=1}^{m}1\{y^{(i)}=0\}n_i + |V|}
\end{align}


详情略, 待补充









		\section{支持向量机}

\subsection{逻辑回归的缺陷}
\begin{enumerate}
	\item 按照逻辑回归的计算方式,我们得到的是一个分隔线(或平面或超平面\footnote{1维称为线,2维称为平面,更多维就称为超平面}),对于我们的数据点,总有一些点离超平面较近(如点C),有一些点离超平面较远(如点A);对于离超平面较远的点,我们有很大把握保证我们的预测是准确的,但是,对于离超平面较近的点我们就没把握了。
	\begin{figure}[htbp]
		\centering
		\includegraphics[scale=0.5]{images/逻辑回归缺陷讲述}
		\caption{支持向量机实例用图}
	\end{figure}
\end{enumerate}

\subsection{SVM前言}
\begin{enumerate}
	\item 在后续介绍SVM过程中,为了描述方便,我们对之前逻辑回归的一些描述做了更改
	\item 将$h_\theta(x)$更改为$h_{w,b}(x)$,其中:$w \to \left[\begin{matrix}\theta_1 \\ \theta_2 \\ \vdots \\ \theta_n  \end{matrix}\right]$,$b \to \theta_0$\footnote{此处为了表达方便,使用$\to$来表达类似的对应关系,且勿将其当做等于"$=$"}
	\begin{align}
		h_{w,b}(x) = g(w^Tx+b)
	\end{align}
	\item 将类标签$y$改为$\{-1, 1\}$;将$h=g(w^T+b)$的值从值域$\{0,1\}$切换到$\{-1, 1\}$,其中,$0\to -1$,$1\to 1$
	,且取消原有的$x_0=1$假设。通过此方式将$h(x)$由参数$\theta$变成了参数$(w, b)$,将截距项$b$与其他项分隔开,以便分析
\end{enumerate}

\subsection{Margin介绍}
\subsubsection{函数间隔}
\begin{enumerate}
	\item 对于数据点$(x^{(i)}, y^{(i)})$,其函数间隔定义为:
	\begin{align}
		\hat{\gamma}^{(i)} = y^{(i)}(w^Tx^{(i)} + b)
	\end{align}
	\item 对于$y^{(i)}=1$的点,为了让函数间隔$\hat{\gamma}$越大,我们需要让$w^Tx^{(i)} + b$正向越大;
	\item 对于$y^{(i)}=-1$的点,为了让函数间隔$\hat{\gamma}$越大,我们需要让$w^Tx^{(i)} + b$负向越大;
	\item 对于给定的训练集$S=\{(x^{(i)}, y^{(i)}); \quad i = 1, \dots, m\}$,我们将其中最小的函数间隔记为$\hat{\gamma}$:
	\begin{align}
		\hat{\gamma} = \min_{i=1,\dots,m}\hat{\gamma}^{(i)}
	\end{align}
	\item 对于函数间隔,若将参数$(w,b)$扩大2倍2为$(2w,2b)$,函数间隔的值$\hat{h}_{w,b}(x)$也变成了原来的2倍,但实际上,这对训练的效果并没有影响,原来错误的现在还是错误
\end{enumerate}

\subsubsection{几何间隔}
\begin{enumerate}
	\item 几何间隔表示为$\gamma^{(i)}$\footnote{与函数间隔相比,少了头顶的帽子: $\hat{ }$}。
	\item 对于数据点$(x^{(i)}, y^{(i)})$,其几何间隔定义为\footnote{证明过程:使用向量的加法表示出超平面外的某点的向量表示形式,然后两边乘以$w^T$,整理一下就能得到几何间隔表示形式,详情略}:
	\begin{align}
		\gamma^{(i)} &= \frac{\hat{\gamma}^{(i)}}{\|w\|}  \\
		&= \frac{y^{(i)}(w^Tx^{(i)} + b)}{\|w\|}  \\
		&=  y^{(i)} \left(\left(\frac{w}{\|w\|}\right)^T x^{(i)} + \frac{b}{\|w\|}\right)
	\end{align}
	\footnote{$\|w\|$是矩阵的范数,$\|w\|^2=w^Tw$}
	\item 当$\|w\|=1$时,几何间隔与函数间隔相等
	\item 相较于函数间隔,几何间隔任意缩放参数$(w,b)$不影响$h_{w,b}(x)$的值;鉴于此性质,我们可以在训练时对$w, b$视需求进行缩放
	\item 点$(x_0, y_0)$到直线$Ax+By+C=0$的距离公式为:
	\begin{align}
		\frac{|Ax_0 + By_0 + C|}{\sqrt{A^2+B^2}}
	\end{align}
	\item 两条平行线$Ax+By+C_1=0$、$Ax+By+C_2=0$的距离公式为:
	\begin{align}
		\frac{|C_1 - C_2|}{\sqrt{A^2+B^2}}
	\end{align}
	\item 同样,对于给定的训练集$S=\{(x^{(i)}, y^{(i)}); \quad i = 1, \dots, m\}$,我们将其中最小的几何间隔记为$\gamma$:
	\begin{align}
		\gamma = \min_{i=1,\dots,m}\gamma^{(i)}
	\end{align}
\end{enumerate}

\subsubsection{函数间隔\&几何间隔}
\begin{enumerate}
	\item 当预测正确时,二者都是正值;当预测错误是,二者都是负值。
\end{enumerate}

\subsubsection{对SVM设计函数间隔、几何间隔原因的思考}
{\color{red}{注意,此部分内容仅仅是我自己的理解,不见得正确,请批判性地看。}}
\begin{enumerate}
	\item 与之前的逻辑回归类似,此处$w^Tx+b=0$是用来判断预测值$h_{w,b}(x)$取1或-1的分界线:若$w^Tx+b\geq0$,则通过激活函数$g(z)$让$h_{w,b}(x)=g(w^Tx+b)=1$,反之,让$h_{w,b}(x)=-1$。$w^Tx+b=0$相当于门槛的角色,没过门槛是一个世界,过了门槛又是另一个世界
	% \item 与前面逻辑回归一样,$w^Tx+b$与$\theta^Tx$计算出来的结果意义一样,且都是通过激活函数$g(z)$(虽然两者的$g(z)$形式不一样)来得到预测值;
	\item 但是,两者的激活函数$g(z)$不一样,其得到的结果也不一样,逻辑回归的$g(z)$得到的是概率值,后续也是通过让概率值达到最大来优化算法;
	\item SVM的$g(z)$得到的结果是一个我也不知道代表什么意义的值,后续优化的对象是预测值与门槛$w^Tx+b=0$的几何间隔
	\item 但是SVM优化的几何间隔又与线性回归中使用点与点间的距离$h_\theta(x^{(i)})-y^{(i)}$不大一样:它除了与线性回归类似计算了与{\color{red}{门槛}}的距离$w^Tx^{(i)}-b-0$\footnote{此处计算的是与{\color{red}{门槛}}的距离,对于门槛,$w^Tx-b$的值为0}之外,还乘以了它的实际值$y^{(i)}$,$y^{(i)}$的出现可以纠正它的正负号,若预测正确(即预测值与实际值在门槛的同一边),则得到的结果为正值;否则为负值,将此定义为函数间隔;又因为让函数间隔最大化不一定能够得到最优的门槛(总有些投机倒把的通过增大$w,b$来达到目的),于是我们又对函数间隔除以$\|w\|$,解决了这个Bug,将此定义为几何间隔。于是对SVM的优化就是对几何间隔最大值的优化\footnote{为什么对线性回归的优化是取最小值,而对SVM的优化是取最大值?这就是我们要能理解每一步推理目的的原因了}。
\end{enumerate}





















			\subsection{最优间隔分类器-优化目标}
\subsubsection{将难以优化的目标转为容易优化的}
\begin{enumerate}
	\item 通过前面的讲述我们知道,为了对给定的数据集进行分类,我们需要找到使得几何间隔最大化的决策边界(在下面的讲述中,先假设我们会遇到的数据集都是线性可切分的,等我们讲到核函数时再去除这个假设)。于是,可以用以下的式子来描述我们要优化的目标
	\begin{align}
		\max_{\gamma, w, b} \gamma
	\end{align}
	根据前面的定义:$\gamma = \min_{i=1,\dots,m}\gamma^{(i)}$,$\gamma$是所有数据点中几何间隔最小值,所以,对于任意数据点,其几何间隔均应大于$\gamma$\footnote{这是约束条件,请注意优化目标与约束条件的差别。}:
	\begin{align}
		y^{(i)}\frac{(w^Tx^{(i)}+b)}{\|w\|} \geq \gamma, \quad i=1, 2, \dots, m
	\end{align}
	因为对于几何间隔,缩放$w$或$b$或都缩放不会影响决策边界的位置\footnote{决策边界始终由$w^Tx^{(i)}+b=0$确定},于是,我们定义$\|w\|=1$,让几何间隔与函数间隔相等,于是:
	\begin{align}
		&\text{优化目标:} \\
		& \qquad \max_{\gamma, w, b} \gamma \\
		&\text{约束条件:} \\
		& \qquad y^{(i)}(w^Tx^{(i)}+b) \geq \gamma, \quad i=1, 2, \dots, m \\
		& \qquad \|w\| = 1
	\end{align}

	\item 经过上面的转化后,$\|w\|=1$这个约束还是不好处理,我们需要想办法去除这个约束,于是将优化目标$\gamma$改写为$\frac{\hat{\gamma}}{\|w\|}$\footnote{课上有人问为什么要使用函数间隔?为什么要对它除以$\|w\|$?实际上$\frac{\hat{\gamma}}{\|w\|}$只是$\gamma$的另一种表述,为了后续消去不好处理的约束$\|w\| = 1$才这样处理的。}:
	\begin{align}
		\max_{\gamma, w, b} \frac{\hat{\gamma}}{\|w\|}
	\end{align}
	约束条件变为:
	\begin{align}
		y^{(i)}\frac{(w^Tx^{(i)}+b)}{\|w\|} \geq \frac{\hat{\gamma}}{\|w\|}, \quad i=1, 2, \dots, m
	\end{align}
	消去$\|w\|$,得:
	\begin{align}
		&\text{优化目标:} \\
		& \qquad \max_{\gamma, w, b} \frac{\hat{\gamma}}{\|w\|} \\
		&\text{约束条件:} \\
		& \qquad y^{(i)}(w^Tx^{(i)}+b) \geq \hat{\gamma}, \quad i=1, 2, \dots, m
	\end{align}
	如上,经过这样的转化,我们将$\|w\|$消掉了,同时$\|w\| = 1$的假设也可以取消了。

	\item 我们令$\hat{\gamma}=1$\footnote{这样做是合理的,因为不论$\hat{\gamma}$取何值,我们都可以通过缩放$\|w\|$来消除实际的$\hat{\gamma}$与$\hat{\gamma}=1$的差距}\footnote{有的地方会将两条经过支持向量的直线标记为$w^Tx+b=1$和$w^Tx+b=-1$,这是正确的,因为,函数间隔中,$y^{(i)}$的作用只是改变函数间隔$\hat{\gamma}$的正负号,而支持向量对应的就是函数间隔为1的点,代入后就可以得到这两条直线的方程},于是优化目标变成:
	\begin{align}
		\max_{\gamma, w, b} \frac{1}{\|w\|}
	\end{align}
	最大化$\frac{1}{\|w\|}$也即最小化$\|w\|$,即最小化$\|w\|^2$,故:
	\begin{align}
		&\text{优化目标:} \\
		& \qquad \min_{\gamma, w, b} \frac{1}{2}\|w\|^2 \\
		&\text{约束条件:} \\
		& \qquad y^{(i)}(w^Tx^{(i)}+b) \geq 1, \quad i=1, 2, \dots, m
	\end{align}
\end{enumerate}




































			\subsection{拉格朗日对偶规划}
\subsubsection{拉格朗日乘数法}
\begin{enumerate}
	\item 在等式约束下求最优问题求解中可使用拉格朗日乘数法,其要解决的问题可表述如下:
	\begin{align}
		&\text{优化目标:} \\
		& \qquad \min_{w} f(w) \\
		&\text{约束条件:} \\
		& \qquad h_i(w) = 0, \quad i=1,\dots,l
	\end{align}
	$l$是等式约束的个数
	\item 其拉格朗日函数为:
	\begin{align}
		\mathcal{L}(w,\beta) = f(w) + \sum_{i=1}^{l}\beta_ih_i(w)
	\end{align}
	\item 求拉格朗日函数的偏导数
	\begin{align}
		\frac{\partial \mathcal{L}}{\partial w_i} &= 0\\
		\frac{\partial \mathcal{L}}{\partial \beta_i} &= 0
	\end{align}
	求解上式,得到$w, \beta$即可
\end{enumerate}

\subsubsection{拉格朗日对偶规划}
\begin{enumerate}
	\item 对于有不等约束的问题,拉格朗日乘数法就无能为力了;这时就需要用拉格朗日对偶规划,其所要解决的问题可表述如下:
	\begin{align}
		&\text{优化目标:} \\
		& \qquad \min_{w} f(w) \\
		&\text{约束条件:} \\
		& \qquad g_i(w) \leq 0, \quad i=1,\dots,k \\
		& \qquad h_i(w) = 0, \quad i=1,\dots,l
	\end{align}
	$k, l$为对应约束的个数。\\
	称其为原问题。
	\item 其Generalized Lagrangian\footnote{{\color{red}{不知道怎么翻译}}}为:
	\begin{align}
		\mathcal{L}(w, \alpha, \beta) = f(w) + \sum_{i=1}^{k}\alpha_ig_i(w) + \sum_{i=1}^{l}\beta_ih_i(w)
	\end{align}
	\item 下面我们定义以下式子:
	\begin{align}
		\theta_{\mathcal{P}}(w) = \max_{\alpha, \beta; \alpha_i\geq0} \mathcal{L}(w, \alpha, \beta)
	\end{align}
	注意,在上面的式子中,我们对$\alpha_i$做了限制:$\alpha_i\geq 0$,于是,$\theta_{\mathcal{P}}(w)$便有了以下性质:
	\begin{enumerate}
		\item 当$g_i(w)$不满足约束条件,即$g_i(w)>0$时,为了使$\mathcal{L}(w, \alpha, \beta)$取到更大,我们只需要让$\alpha_i$更大就行,很显然,此时$\max \mathcal{L}(w, \alpha, \beta) = +\infty = \theta_{\mathcal{P}}(w)$
		\item 当$h_i(w)$不满足约束条件时,即$h_i(w)>0$或$h_i(w)<0$,我们同样只需要更改$\beta_i$的值就能让$\max \mathcal{L}(w, \alpha, \beta)$取到$+\infty$
		\item 当$g_i(w)$与$h_i(w)$均满足约束条件时,$\sum_{i=1}^{k}\alpha_i g_i(w)$的最大值是0;$\sum_{i=1}^{l}\beta_ih_i(w)$始终为0;所以,$\theta_{\mathcal{P}}(w)=\max_{\alpha, \beta; \alpha_i\geq0} \mathcal{L}(w, \alpha, \beta)=f(w)$
		\item 综上,我们可以得到以下式子:
		\[ \theta_{\mathcal{P}}(w)=\begin{cases}
		f(w) \quad \text{符合所有约束条件} \\
		+\infty \quad \text{其他}
		\end{cases} \]
	\end{enumerate}

	\item 通过上面的式子我们可以知道,在满足所有约束条件的情况下,$f(w)=\theta_{\mathcal{P}}(w)$,求解$f(w)$的最小值就等同于求解$\theta_{\mathcal{P}}(w)$的最小值(所以我们将$\theta_{\mathcal{P}}(w)$称为{\color{blue}{原问题}}),于是我们的优化目标就变成了:
	\begin{align}
		\min_{w} \max_{\alpha, \beta; \alpha_i \geq 0} \mathcal{L}(w, \alpha, \beta)
	\end{align}
	注意:
	\begin{enumerate}
		\item 对于$\min_{w}$,这是由我们的优化目标$\min_{w}f(w)$带来的
		\item $\max_{\alpha, \beta; \alpha_i \geq 0} \mathcal{L}(w, \alpha, \beta)$是一个整体,此式子表示的是取得$\mathcal{L}(w, \alpha, \beta)$的最大值的最小值
	\end{enumerate}

	\item 相对于前面的求拉格朗日函数最大值$\theta_{\mathcal{P}}(w) = \max_{\alpha, \beta; \alpha_i\geq0} \mathcal{L}(w, \alpha, \beta)$,其{\color{blue}{对偶问题}}\footnote{为什么要讲到其对偶问题呢?后面会讲到。}为求拉格朗日函数的最小值:
	\begin{align}
		\theta_{\mathcal{D}}(\alpha, \beta) = \min_{w} \mathcal{L}(w, \alpha, \beta)
	\end{align}
	对偶问题对应的{\color{blue}{对偶优化问题}}为:
	\begin{align}
		\max_{\alpha, \beta;\alpha_i \geq 0} \theta_{\mathcal{D}}(\alpha, \beta) = \max_{\alpha, \beta;\alpha_i \geq 0} \min_{w} \mathcal{L}(w, \alpha, \beta)
	\end{align}
	\footnote{注意,这个对偶优化问题并不是我们推导出来的,可以说是我们为了后续需要设计出来的,所以不要去想为什么这个式子是这样子的了。}

	\item 对比一下原问题与其对偶问题,可以发现,两者之间的差别只是对调了求最大值$\max_{\alpha, \beta;\alpha_i \geq 0}$与最小值$\min_{w}$的顺序。在后面的描述中,我们用$p^{*}$表示原问题的最优解,用$d^{*}$表示对偶问题的最优解:
	\begin{align}
		p^{*} &= \min_{w} \max_{\alpha, \beta; \alpha_i \geq 0} \mathcal{L}(w, \alpha, \beta) \\
		d^{*} &= \max_{\alpha, \beta;\alpha_i \geq 0} \min_{w} \mathcal{L}(w, \alpha, \beta)
	\end{align}

	\item 原问题与其对偶问题有以下性质:
	\begin{align}
		d^{*} = \max_{\alpha, \beta;\alpha_i \geq 0} \min_{w} \mathcal{L}(w, \alpha, \beta) \leq 
		\min_{w} \max_{\alpha, \beta; \alpha_i \geq 0} \mathcal{L}(w, \alpha, \beta) = p^{*}
	\end{align}
	在一些特定条件(KKT条件)下,取等号。这时,就可以将原优化问题转化为对偶优化问题求解了。

	\item KKT条件 \\
	假设$f(w), g_i(w)$是凸函数、$h_i(w)$是仿射函数、存在$w$使得$g_i(w) < 0$对所有的$i$都成立,那么一定会存在$w^*, \alpha^*, \beta^*$\footnote{其中$w^*$是原问题的解,$\alpha^*, \beta^*$是对偶问题的解,且$p^*=d^*=\mathcal{L}(w^*, \alpha^*, \beta^*)$}满足以下条件:
	\begin{align}
		\frac{\partial\mathcal{L}(w^*, \alpha^*, \beta^*)}{\partial w_i} &= 0, \quad i=1, \dots, n \\
		\frac{\partial\mathcal{L}(w^*, \alpha^*, \beta^*)}{\partial \beta_i} &= 0, \quad i=1, \dots, l \\
		\alpha_i^*g_i(w^*) &= 0, \quad i=1, \dots, k \\
		g_i(w^*) &\leq 0, \quad i=1, \dots, k  \\
		\alpha^* &\geq 0, \quad i=1, \dots, k 
	\end{align}
	其中:\\
	$\alpha_i^*g_i(w^*) = 0, \quad i=1, \dots, k$称为{\color{blue}{KKT对偶互补条件}};当$\alpha^* > 0$时,$g_i(w^*) = 0$。后续会用到此性质。
\end{enumerate}
本部分内容参考资料:\url{http://blog.csdn.net/Victor_Gun/article/details/45227999}









































			\subsection{最优间隔分类器-优化方法}
\begin{enumerate}
	\item 根据前面推导出来的优化目标及拉格朗日对偶规划的内容,我们可以得到不等式约束为:
	\begin{align}
		g_i(w) = -y^{(i)}\left(w^Tx^{(i)}+b\right)+1 \leq 0
	\end{align}
	无等式约束。\\
	

\end{enumerate}


















			\subsection{核}
为方便后续内容的理解,我先大体讲一下下面的内容: \\

在我们碰到的问题中,有些在低维下不是线性可分的,但是其在高维下却是线性可分的,于是,我们可以将低维下的特征映射到高维中,然后再高维下进行分割,这里的从低维映射到高维就是的方法就是我们后面会提到的$\phi(x)$;但是,这又会带来一个问题,那就是,映射到高维后,如果仅仅制作映射而不做其他优化,我们计算的时间复杂度就上去了,于是我们就想办法来对计算方法进行改善,这个改善的方法就是使用核函数!通过使用核函数,我们并不需要知道我们从低维映射到高维到底是如何映射的(即我们并不显示地知道$\phi(x)$长什么样,我们只知道通过核函数可以达到与通过$\phi(x)$映射后一样的结果)。 \\

简而言之,引出$\phi(\cdot)$是为了说明低维不可分割的数据在高维可能可以分割;再举一堆例子说明通过得到$\phi(\cdot)$后再求$\phi(\cdot)$内积的方式计算量太大,不可取,有更好的计算方式,那就是使用核函数$K(x, z)$\\

按我个人的经验,吴恩达的视频在你还搞不明白核函数的内容时都听不懂他在讲什么,但是在搞明白后就觉得讲得很清晰(也不排除我没认真听......),可以先参考下下面的这篇博文:\url{http://blog.pluskid.org/?p=685} \\

下面开始正文

\subsubsection{特征映射与核函数引入}
\begin{enumerate}
	\item 在前面的线性回归中,我们提到,为了得到更好的拟合结果,我们可以添加高阶项如$x^2, x^3$等来优化拟合结果。为了区分远有的$x$,以及我们添加的$x^2, x^3, ...$,我们将原有的$x$称为问题的{\color{blue}{属性}},将新添加的$x^2, x^3,...$以及$x$一起称为问题的{\color{blue}{特征}}。从$x \to \left[\begin{matrix}x \\ x^2 \\ x^3\end{matrix}\right]$称为特征映射。记为$\phi$,如本例子中:$\phi(x) = \left[\begin{matrix}x \\ x^2 \\ x^3\end{matrix}\right]$

	\item 在前面的算法中,我们可以将内积$\langle x, z \rangle$改为$\langle \phi(x), \phi(x) \rangle$,我们将其定义为高斯函数:
	\begin{align}
		K(x, z) = \phi(x)^T \phi(z)
	\end{align}

	\item 但是,在映射后$\phi(x)^T \phi(z)$的计算量太大,比如
	\begin{enumerate}
		\item 我们先定义一个核函数$K(x, z)=(x^T z)^2$,然后找出其特征映射$\phi(x)$,以此说明若要显式地写出$\phi(x)$的式子与只知道核函数不使用显式的$\phi(x)$两者间的计算量差异。
		\item 对$K(x, z)$进一步计算
		\begin{align}
			K(x, z) &= \left(\sum_{i=1}^{n}x_iz_i \sum_{j=1}^{n}x_jz_j \right) \\
			&= \sum_{i=1}^{n} \sum_{j=1}^{n} x_i x_j z_i z_j \\
			&= \sum_{i,j=1}^{n}(x_i x_j) (z_i z_j)
		\end{align}
		如上,以$n=3$为例,若要将$K(x, z)$表示成$\phi(x)^T\phi(z)$的形式:
		\begin{align}
			\phi(x) = \left[\begin{matrix}x_1 x_1 \\ x_1 x_2 \\ x_1 x_3 \\ x_2 x_1 \\ x_2 x_2 \\ x_2 x_3 \\ x_3 x_1 \\ x_3 x_2 \\ x_3 x_3 \\\end{matrix}\right]
		\end{align}
		$\phi(z)$同理。显然,在上面的$\phi(x)$中,若要讲$x \to \phi(x)$,我们需要进行9次的计算;
		\item 但是,若我们不管$\phi(x)$应如何表示,直接先计算$x^Tz$后再取平方,$n=3$时只需要计算3次。
		\item 实际上,此处的$K(x, z) = (x^T z)^2$ 是我们后面会讲到的多项式核$K(x,z)=\left(\langle x, z \rangle +R \right)^d = \left(x^Tz +R\right)^d$的一种。虽然核函数$K(x,z)$可以很容易地表示出来,但是其特征映射$\phi(x)$却不见得容易表示。
	\end{enumerate}

	\item 所以,虽然前面使用将$x$映射到高维$\phi(x)$来说明这样做可以将低维无法线性分隔的点分隔开,但是在实际计算中,我们并不会去计算$\phi(x)$应如何表示,相当于我们实际上使用更容易计算的核函数$K(x, z)$来达到特征映射$\phi(x)$将低维映射到高维的效果

	\item 我们会做的是找到一个核函数,证明它确实对应着存在一个特征映射$\phi(x)$可以实现将数据在高维特征下分割开。

	\item 注: 下面说明下\url{http://blog.pluskid.org/?p=685}中的例子应如何理解
	\begin{enumerate}
		\item 例子中说明提到:$\phi(\cdot)$的目的是使得向量$x_1=(\eta_1, \eta_2)^T, x_2=(\xi_1, \xi_2)^T$在经过$\phi(\cdot)$映射后再求内积的结果为:
		\begin{align}
			\langle \phi(x_1), \phi(x_2) \rangle = \eta_1\xi_1 + \eta_1^2\xi_1^2 + \eta_2\xi_2 + \eta_2^2\xi_2^2 + \eta_1\eta_2\xi_1\xi_2
		\end{align}
		\item 但是,若我们不知道$\phi(\cdot)$是何种形式,直接通过某个核函数$K(x_1, x_2)$也可以得到同样的结果:
		\begin{align}
			\left( \langle x_1, x_2 \rangle +1 \right)^2 &= \langle x_1, x_2 \rangle ^2 + 2 \langle x_1, x_2 \rangle + 1 \\
			&= (x_1^T x_2)^2 + 2x_1^T x_2 + 1 \\
			&=\left\{ \left[\begin{matrix}\eta_1 & \eta_2\end{matrix}\right]\left[\begin{matrix}\xi_1 \\ \xi_2\end{matrix}\right] \right\}^2 + 2\left[\begin{matrix}\eta_1 & \eta_2\end{matrix}\right]\left[\begin{matrix}\xi_1 \\ \xi_2\end{matrix}\right] + 1 \\
			&= (\eta_1\xi_1+\eta_2\xi_2)^2 + 2(\eta_1\xi_1+\eta_2\xi_2) + 1 \\
			&= \eta_1^2\xi_1^2 + \eta_2^2\xi_2^2 + 2\eta_1\eta_2\xi_1\xi_2 + 2\eta_1\xi_1 + 2\eta_2\xi_2 + 1 \\
		\end{align}
		与前面的式子相比,只要对某些维度进行线性缩放即可,说明两者可以达到同样的效果。
		\item 但是,在计算量上,用第一种方法我们需要先找到其对应的$\phi(\cdot)$(若$\phi(x_1, x-2)=(\sqrt{2}x_1, x_1^2, \sqrt{2}x_2, x_2^2, \sqrt{2}x_1x_2, 1)^T$则可得到与$\left( \langle x_1, x_2 \rangle +1 \right)^2$一样的结果),然后再计算$\phi(\cdot)$的内积。显然,通过$\phi(\cdot)$的方式计算量太大,但是使用核函数计算的方式计算量就小多了。这就是该例子的目的
	\end{enumerate}

\end{enumerate}

\subsubsection{核函数需要满足的条件}


\subsubsection{不同的核函数介绍}






















		\include{CS299课程笔记/contents/附录/概念与定义}
			\subsection{中英对照表}
\begin{enumerate}
	\item 假设函数 - Hypothesis Function
	\item 学习速率 - Learning Rate
	\item 最小均方 - Least Mean Squares - LMS
	\item 线性回归 - Linear Regression - LR
	\item 逻辑回归 - Logistic Regression - LR
	\item 梯度下降 - Gradient Descent - GD
	\item 随机梯度下降 - Stochastic Gradient Descent - SGD
	\item 局部加权回归 - Local Weight Regression - LWR
	\item 独立同分布 - Independently and Identically Distributed - IID
	\item 广义线性模型 - Generalized Linear Model - GLM
	\item 正则响应函数 - Canonical Response Function
	\item 正则关联函数 - Canonical Link Function
	\item 判别学习算法 - Discriminative Learning Algorithm - DLA
	\item 生成学习算法 - Generative Learning Algorithm - GLA
	\item 朴素贝叶斯 - Naive Bayes
	\item 拉普拉斯平滑 - Laplace Smoothing
	\item  - Generalization Error
	\item 支持向量机 - Support Vector Machine - SVM
	\item 函数间隔 - Functional Margin
	\item 几何间隔 - Geometric Margins
	\item 序列最小优化 - Sequential Minimal Optimization - SMO
	\item 偏差 - Bias
	\item 方差 - Variance
	\item 经验风险最小化 - Empirical Risk Minization - ERM
	\item 联合界 - Union Bound
	\item 一致收敛 - Uniform Convergence
	\item 交叉验证 - Cross Validation
	\item 留一交叉验证 - Leave-One-Out Cross Validation
	\item 向前搜索 - Forward Search
	
\end{enumerate}
			\subsection{思考}
{\color{red}{以下内容均为个人的思考,不一定是正确的。}}
\begin{enumerate}
	\item 机器学习主要就是围绕几个部分进行各种改进:
	\begin{enumerate}
		\item 假设函数:$h_\theta(x)$
		\item Cost Function: $J(\theta)$
			\begin{enumerate}
				\item Cost Function仅仅只是一种评价预测值与实际值的方式,就算换种评价方式,差距大的仍旧差距大。表面上看起来通过改进Cost Function并没法得到多少改进,但实际上并非如此
				\item 一方面,通过改进Cost Function,可以在计算预测值与实际值误差时给不同的数据点$(x^{(i)}, y^{(i)})$不同的权重,让较重要的点的有较大权重,较不重要的点权重较小。事实上,局部加权回归就是采用此方式
				\item 另一方面,计算Cost Fuction也是需要计算量的,若采用更简单的计算方式也算对算法的一种优化。{\color{gray}{暂未找到实例}}
			\end{enumerate}
		\item 获取最优解的方式,如梯度下降中使用$\theta_j := \theta_j + \frac{\partial J(\theta)}{\theta_j}$进行迭代

	\end{enumerate}
\end{enumerate}


















	\part{Coursera机器学习课程笔记}
		% \title{CS299课程笔记}
\author{MingShun Wu}
\date{\today}
\maketitle
		% \renewcommand{\contentsname}{目录}
\tableofcontents
\setcounter{tocdepth}{3}
		\include{Coursera机器学习课程笔记/contents/Linear_Regression}
		\section{逻辑回归(Logistic Regression)}
\subsection{当只有2个类别时,使用1个分类器}

\subsubsection{sigmoid函数}
\begin{equation}
	sigmoid(z) = \frac{1}{1 + e^z}
\end{equation}


\subsubsection{预测函数}
\begin{enumerate}
\item 数值形式
\begin{equation}
	h_\theta(x) = \frac{1}{1 + e^{\theta^T x}}
\end{equation}

\item 矩阵形式
\begin{equation}
	h_\theta(X) = \frac{1}{1 + e^{X \theta}}
\end{equation}
\end{enumerate}


\subsubsection{Cost Function}
\begin{enumerate}
\item 数值形式
\begin{equation}
	J(\theta) = \frac{1}{m}
	    \sum_{i=1}^m \left[ -y^{(i)}log{h_\theta(x^{(i)})} - (1-y^{(i)})log{(1-h_\theta(x^{(i)}))} \right]
\end{equation}

\item 矩阵形式
\begin{equation}
		J(\theta) = \frac{1}{m} \left[-y^T \log{h_\theta(x)} - (1-y^T) \log{(1-h_\theta(x)}\right]
\end{equation}
\end{enumerate}

\subsubsection{偏导数$\frac{\partial J(\theta)}{\partial \theta_j}$}
\begin{enumerate}
\item 数值计算形式
\begin{equation}
	\frac{\partial J(\theta)}{\partial \theta_j} =
	    \frac{1}{m} \sum_{i=1}^m \left[h_\theta(x^{(i)}) - y^{(i)}\right] x_j^{(i)}
\end{equation}

\item 矩阵计算形式
\begin{equation}
	\nabla J(\theta) = \frac{1}{m} X^T \left[h_\theta(x) - y\right]
\end{equation}
\end{enumerate}


\subsubsection{梯度下降迭代算法}
\begin{enumerate}
\item 数值计算形式
\begin{equation}\begin{aligned}
	\theta_j &:= \theta_j - \alpha\frac{\partial J(\theta)}{\partial \theta_j} \\
	    &:= \theta_j - \alpha \frac{1}{m} \sum_{i=1}^m \left[h_\theta(x^{(i)}) - y^{(i)}\right] x_j^{(i)}
\end{aligned}\end{equation}

\item 矩阵计算形式
\begin{equation}\begin{aligned}
	\theta &:= \theta - \alpha\nabla J(\theta) \\
		&:= \theta - \alpha \frac{1}{m} X^T \left[h_\theta(x) - y\right]
\end{aligned}\end{equation}
\end{enumerate}



\subsection{当只有k个类别时,使用k个分类器}
% 当只有k个类别时,使用k个分类器

		\include{Coursera机器学习课程笔记/contents/Regularization}
		\include{Coursera机器学习课程笔记/contents/Neural_Networks_Forward_legend}
		\include{Coursera机器学习课程笔记/contents/Neural_Networks_Forward}
		\section{神经网络--后向算法}

\subsection{神经网络示意图--后向算法}
\begin{tikzpicture}
\tikzset{
	a/.style={circle, draw=black}
}

% x
% \node[a] (x_0) at(0,10) {$x_0=+1$};
\node[a] (x_1) at(0,8) {$x_1$};
\node[a] (x_2) at(0,6) {$x_2$};
\node[a] (x_3) at(0,4) {$x_3$};
\filldraw (0,3) circle (.06);
\filldraw (0,2) circle (.06);
\filldraw (0,1) circle (.06);
\node[a] (x_n) at(0,0) {$x_n$};

% X与a的分隔线
\draw [dashed] (1,-1) -- (1,9);

% \delta^{(1)}
\node[a] (a_10) at(2,10)   {0};
\node[a] (a_11) at(2,8)   {0};
\node[a] (a_12) at(2,6) {0};
\node[a] (a_13) at(2,4)   {0};
\filldraw (2,3) circle (.06);
\filldraw (2,2) circle (.06);
\filldraw (2,1) circle (.06);
\node[a] (a_1s1) at(2,0) {0};

% \delta^{(2)}
\node[a] (a_20) at(6,9.5) {$\delta_0^{(2)}$};
\node[a] (a_21) at(6,8.0) {$\delta_1^{(2)}$};
\node[a] (a_22) at(6,6.5) {$\delta_2^{(2)}$};
\node[a] (a_23) at(6,5.0) {$\delta_3^{(2)}$};
\node[a] (a_24) at(6,3.5) {$\delta_4^{(2)}$};
\filldraw (6,2.4) circle (.06);
\filldraw (6,1.8) circle (.06);
\filldraw (6,1.2) circle (.06);
\node[a] (a_2s2) at(6,0) {$\delta_{s_2}^{(2)}$};

% \delta^{(3)}
\node[a] (a_30) at(10,9.5) {$\delta_0^{(3)}$};
\node[a] (a_31) at(10,8.0) {$\delta_1^{(3)}$};
\node[a] (a_32) at(10,6.5) {$\delta_2^{(3)}$};
\node[a] (a_33) at(10,5.0) {$\delta_3^{(3)}$};
\node[a] (a_34) at(10,3.5) {$\delta_4^{(3)}$};
\filldraw (10,2.4) circle (.06);
\filldraw (10,1.8) circle (.06);
\filldraw (10,1.2) circle (.06);
\node[a] (a_3s3) at(10,0) {$\delta_{s_3}^{(3)}$};

% 中间部分
\filldraw (12,0) circle (.06);
\filldraw (12,2) circle (.06);
\filldraw (12,4) circle (.06);
\filldraw (12,6) circle (.06);
\filldraw (12,8) circle (.06);

% \delta^{(n)}

\node[a] (a_L1) at(14,7.5) {$\delta_1^{(L)}$};
\node[a] (a_L2) at(14,6)   {$\delta_2^{(L)}$};
\node[a] (a_L3) at(14,4.5) {$\delta_3^{(L)}$};
\node[a] (a_L4) at(14,3)   {$\delta_4^{(L)}$};
\filldraw (14,2.0) circle (.06);
\filldraw (14,1.5) circle (.06);
\filldraw (14,1.0) circle (.06);
\node[a] (a_Lsl) at(14,0) {$\delta_{s_l}^{(L)}$};


% % 后向算法图例--仅连线:\delta21 -- \delta_1
% \draw[->] (a_21) -- (a_10);
% \draw[->] (a_21) -- (a_11);
% \draw[->] (a_21) -- (a_12);
% \draw[->] (a_21) -- (a_13);
% \draw[->] (a_21) -- (a_1s1);
% % 后向算法图例--仅连线:\delta22 -- \delta_1
% \draw[->] (a_22) -- (a_10);
% \draw[->] (a_22) -- (a_11);
% \draw[->] (a_22) -- (a_12);
% \draw[->] (a_22) -- (a_13);
% \draw[->] (a_22) -- (a_1s1);
% % 后向算法图例--仅连线:\delta23 -- \delta_1
% \draw[->] (a_23) -- (a_10);
% \draw[->] (a_23) -- (a_11);
% \draw[->] (a_23) -- (a_12);
% \draw[->] (a_23) -- (a_13);
% \draw[->] (a_23) -- (a_1s1);
% % 后向算法图例--仅连线:\delta24 -- \delta_1
% \draw[->] (a_24) -- (a_10);
% \draw[->] (a_24) -- (a_11);
% \draw[->] (a_24) -- (a_12);
% \draw[->] (a_24) -- (a_13);
% \draw[->] (a_24) -- (a_1s1);
% % 后向算法图例--仅连线:\delta2s2 -- \delta_1
% \draw[->] (a_2s2) -- (a_10);
% \draw[->] (a_2s2) -- (a_11);
% \draw[->] (a_2s2) -- (a_12);
% \draw[->] (a_2s2) -- (a_13);
% \draw[->] (a_2s2) -- (a_1s1);


% 后向算法--仅连线: \delta31 --> \delta2
\draw[->] (a_31) -- (a_20);
\draw[->] (a_31) -- (a_21);
\draw[->] (a_31) -- (a_22);
\draw[->] (a_31) -- (a_23);
\draw[->] (a_31) -- (a_24);
\draw[->] (a_31) -- (a_2s2);
% 后向算法--仅连线: \delta32 --> \delta2
\draw[->] (a_32) -- (a_20);
\draw[->] (a_32) -- (a_21);
\draw[->] (a_32) -- (a_22);
\draw[->] (a_32) -- (a_23);
\draw[->] (a_32) -- (a_24);
\draw[->] (a_32) -- (a_2s2);
% 后向算法--仅连线: \delta33 --> \delta2
\draw[->] (a_33) -- (a_20);
\draw[->] (a_33) -- (a_21);
\draw[->] (a_33) -- (a_22);
\draw[->] (a_33) -- (a_23);
\draw[->] (a_33) -- (a_24);
\draw[->] (a_33) -- (a_2s2);
% 后向算法--仅连线: \delta34 --> \delta2
\draw[->] (a_34) -- (a_20);
\draw[->] (a_34) -- (a_21);
\draw[->] (a_34) -- (a_22);
\draw[->] (a_34) -- (a_23);
\draw[->] (a_34) -- (a_24);
\draw[->] (a_34) -- (a_2s2);
% 后向算法--仅连线: \delta3s3 --> \delta2
\draw[->] (a_3s3) -- (a_20);
\draw[->] (a_3s3) -- (a_21);
\draw[->] (a_3s3) -- (a_22);
\draw[->] (a_3s3) -- (a_23);
\draw[->] (a_3s3) -- (a_24);
\draw[->] (a_3s3) -- (a_2s2);


% 标志
\node at (4, 11) {$\Theta^{(1)}$};
\node at (8, 11) {$\Theta^{(2)}$};
\node at (2, -1) {\color{red}{无$\delta^{(1)}$}};
\node at (6, -1) {\color{red}{$\delta^{(2)}$}};
\node at (10, -1) {\color{red}{$\delta^{(3)}$}};


% \node at (4, -2) {\color{red}{$\delta^{(1)}=(\Theta^{(1)})^T [\delta^{2}(2:end)] .* g(z^{1}) .* (1-g(z^{1}))$}};
% \node at (8, -3) {\color{red}{$\delta^{(l)}=(\Theta^{(l)})^T [\delta^{l+1}(2:end)] .* g(z^{l}) .* (1-g(z^{l}))$}};


\end{tikzpicture}



\[\begin{cases}
	\delta^{L} &= a^{L} - y ,\quad l=L \\
	\delta^{L-1} &= (\Theta^{(L-1)})^T \delta^{L} .* g^{'}(z^{L-1}) \\
		&= (\Theta^{(L-1)})^T \delta^{L} .* g(z^{L-1}) .* (1-g(z^{L-1})),\quad l=L-1 \\
		\delta^{l} &= (\Theta^{(l)})^T [\delta^{l+1}(2:end)] .* g^{'}(z^{l}) \\
	&= (\Theta^{(l)})^T [\delta^{l+1}(2:end)] .* g(z^{l}) .* (1-g(z^{l})) ,\quad 2<=l<=L-2 \\
	% 无 a^{(1)}, \quad l=1
\end{cases}\] \\
无 $\delta^{(1)}$

\begin{equation}\begin{aligned}
		\Delta^{(l)} := \Delta^{(l)} + \delta^{(l+1)} (a^{(l)})^T
	\end{aligned}\end{equation}

\[ D_{ij}^{(l)} = \begin{cases}
	\frac{1}{m}\Delta_{ij}^{(l)} ,\quad j=0 \\
	\frac{1}{m}(\Delta_{ij}^{(l)} + \Theta_{ij}^{(l)}), \quad j \neq 0
\end{cases}\]

\begin{equation*}
	\frac{\partial{J(\Theta)}}{\partial{\Theta_{ij}^{(l)}}} = D_{ij}^{(l)}
\end{equation*}





		\include{Coursera机器学习课程笔记/contents/Neural_Networks_Backward}
		\include{Coursera机器学习课程笔记/contents/调试技巧}
		\include{Coursera机器学习课程笔记/contents/svm}

	\part{概率统计基础知识}
		% \title{CS299课程笔记}
\author{MingShun Wu}
\date{\today}
\maketitle
		% \renewcommand{\contentsname}{目录}
\tableofcontents
\setcounter{tocdepth}{3}
		\section{随机事件和概率}
\subsection{基本概念及公式}
\begin{enumerate}
	\item 完备事件组 \\
	满足$\cup_{i=1}^nB_i=\Omega$,且$B_iB_j = \emptyset$的事件组$B_1$, $B_2$, ..., $B_n$成为完备事件组
	\item 乘法公式
	\begin{align}
	P(A_1A_2...A_n) &= P(A_1)P(A_2|A_1)P(A_3|A_1A_2)....P(A_n|A_1A_2...A_{n-1})
	\end{align}
	乘法公式并没有$A_k$要相互独立或条件独立的要求。

	\item 全概率公式
	\begin{equation}
	P(A) = \sum_{i=1}^{n}P(A|B_i)P(B_i)
	\end{equation}
	其中,$B$为一完备事件组

	\item 贝叶斯公式
	\begin{align}
	P(B_j|A) &= \frac{P(A|B_j)P(B_j)}{P(A)} \\
	&= \frac{P(A|B_j)P(B_j)}{\sum_{i=1}^{n}P(A|B_i)P(B_i)},  \quad j = 1, 2, 3, ..., n
	\end{align}
	\footnote{$P(B_j|A)=\frac{P(AB_j)}{P(A)}$,再将分子用乘法公式展开,分母用全概率公式展开,即可得到贝叶斯公式} \\
	同样,要求B为完备事件组。 \\
	贝叶斯公式求得是条件概率,求解过程中一般需要用到全概率公式(除非$P(A)$已知)。 \\
	事情还没有发生,要求这件事情发生的可能性的大小,是先验概率;事情已经发生,要求这件事情发生的原因是由某个因素引起的可能性的大小,是后验概率。所以贝叶斯公式求的是后验概率。
	\item 其他
	\begin{enumerate}
		\item 加法公式
		\begin{align}
			P(A\cup B) &= P(A) + P(B) - P(AB) \\
			P(A\cup B\cup C) &= P(A)+P(B)+P(C)-P(AB)-P(BC)-P(AC) +P(ABC)
		\end{align}
		\item 减法公式
		\begin{equation}
			P(A-B) = P(A)-P(AB)
		\end{equation}
	\end{enumerate}
\end{enumerate}

		\section{常用分布}

\subsection{离散型}
\begin{enumerate}
	\item $0-1$分布
	\begin{align}
		P(X=1) &= p \\
		P(X=0) &= 1-p
	\end{align}
	即
	\begin{align}
		P(X=x) = p^x(1-p)^{(1-x)}
	\end{align}
	\footnote{在机器学习中常用此式子表示$0-1$分布}
	\item 二项分布
	\begin{equation}
		P(X=k) = C_n^kp^kq^{n-k}, \quad k=0, 1, 2, \dots, n, \quad q=1-p
	\end{equation}

	\item 几何分布:第$k$次首次发生
	\begin{equation}
		P(X=k) = pq^{k-1}, \quad k=1, 2, 3, ...
	\end{equation}

	\item 超几何分布:共$N$件产品,其中有$M$件次品,共取$n$次,取到$k$件次品
	\begin{equation}
	P(X=k) = \frac{C_M^kC_{N-M}^{n-k}}{C_N^n}, \quad k = l_1, \dots, l_2
	\end{equation}
	其中,$l_1 = max(0, n-N+M\footnote{通过$N-M {\geq} n-k$可以得到})$, $l_2=min(M,n)$

	\item 泊松分布
	\begin{equation}
		P(X=k) = \frac{\lambda^k}{k!}e^{-\lambda}, \quad k=0, 1, 2, \dots
	\end{equation}
	其中,$\lambda>0$且为常数
\end{enumerate}

\subsection{连续型}\footnote{在数学解题中常用分布函数$F(x)$,若有需要再求导得到概率密度;但在机器学习中,常用概率密度函数$f(x)$}
\begin{enumerate}
	\item 连续型随机变量的分布函数$F(x)$必定连续,但是,其概率密度函数$f(x)$不一定连续
	\item 均匀分布
	\[ f(x)=\begin{cases}
	\frac{1}{b-a}, \quad a \leq x \leq b, \\
	0, \quad Others
	\end{cases} \]

	\item 指数分布
	\[ f(x)=\begin{cases}
	\lambda e^{-\lambda x}, \quad x>0, \\
	0, \quad x \leq 0
	\end{cases} \]
	\footnote{对比一下机器学习中的指数分布族:$p(y;\eta) = b(y)e^{\eta^T T(y) - a(\eta)}$,相当于$b(y)=\lambda, \eta^T=-\lambda, T(y)=y, a(\eta)=0$}
	其中,$\lambda>0$。\\
	其分布函数为
	\[ F(x)=\begin{cases}
	1-e^{-\lambda x}, \quad x>0 \\
	0, \quad x \leq 0
	\end{cases} \]

	\item 正态分布
	\begin{equation}
		f(x) = \frac{1}{\sqrt{2\pi} \sigma}e^{-\frac{(x-\mu)^2}{2\sigma^2}}, \quad -\infty < x < +\infty
	\end{equation}
	其分布函数为
	\begin{equation}
		F(x) = \frac{1}{\sqrt{2\pi}\sigma} \int_{-\infty}^x  e^{-\frac{(t-\mu)^2}{2\sigma^2}} dt, \quad -\infty < x < +\infty
	\end{equation}
	标准正态分布
	\begin{equation}
		\Phi(x) = \frac{1}{\sqrt{2\pi}} \int_{-\infty}^x e^{-\frac{t^2}{2}} dt
	\end{equation}
\end{enumerate}

\subsection{泊松定理}
在伯努利实验中,若$n$较大,$p_n$较小,是的$np_n$在$n \to +\infty$时为一常数,则有
\begin{equation}
	\lim_{n \to +\infty}C_n^k p^k (1-p)^{n-k} = \frac{\lambda ^k}{k!}e^{-\lambda}
\end{equation}
其中,$\lim_{n\to+\infty}np = \lambda$







		\section{多维随机变量及其分布}
\subsection{二维随机变量及其分布}
\subsubsection{连续型}
\begin{enumerate}
	\item 二维随机变量的分布
	\begin{equation}
		F(x, y) = P(X\leq x, Y \leq y), \quad -\infty < x < +\infty, -\infty < y < + \infty
	\end{equation}

	\item 二维随机变量的边缘分布
	\begin{equation}\begin{aligned}
			F_X(x) &= P(X\leq x) = P(X\leq x, y< +\infty) = F(x, +\infty) \\
			F_Y(y) &= P(Y\leq y) = P(X< +\infty, Y\leq y) = F(+\infty, y)
	\end{aligned}\end{equation}
	\item 二维随机变量的条件分布
	\begin{align}
		F_{X|Y}(x|y) = P(X \leq x | Y=y) 
		= \lim_{\epsilon \to 0^+}P(X \leq x | y - \epsilon < Y \leq y + \epsilon) \\
		= \lim_{\epsilon \to 0^+} \frac{P(X \leq x, y - \epsilon < Y \leq y + \epsilon)}{P(y - \epsilon < Y \leq y + \epsilon)}
	\end{align}
	$F_{Y|X}(y|x)$同理 \\
	若$f(x,y)$在$(x,y)$点连续,且$f_Y(y)>0$,则:
	\begin{equation}
		F_{X|Y}(x|y) = \int_{-\infty}^{x}\frac{f(s,y)}{f_Y(y)}ds
	\end{equation}
	$F_{Y|X}(y|x)=\int_{-\infty}^{y}\frac{f(x,s)}{f_X(x)}ds$

\end{enumerate}

\subsubsection{离散型}
\begin{enumerate}
	\item 二维离散型随机变量的概率分布(或称分布律)
	\begin{equation}
		P(X=x_i, Y=y_j) = p_{ij}, \quad i,j = 1, 2, \dots
	\end{equation}

	\item 二维离散型随机变量的边缘分布
	\begin{align}
		p_{i\cdot} &= P(X=x_i) 
		= \sum_{j=1}^{+\infty}P(X=x_i, Y=y_j) 
		= \sum_{j=1}^{+\infty}p_{ij}, \quad i = 1, 2, \dots \\
		p_{\cdot j} &= P(Y=y_j) 
		= \sum_{i=1}^{+\infty}P(X=x_i, Y=y_j) 
		= \sum_{i=1}^{+\infty}p_{ij}, \quad j = 1, 2, \dots 
	\end{align}
	\item 二维离散型随机变量的条件分布
	\begin{equation}\begin{aligned}
		P(X=x_i|Y=y_j) = \frac{P(X=x_i, Y=y_j)}{P(Y=y_j)} = \frac{p_{ij}}{p_{\cdot j}}, \quad i = 1, 2, \dots
	\end{aligned}\end{equation}


\end{enumerate}

\subsection{概率密度}
\subsubsection{连续型}
\begin{enumerate}
	\item 二维随机变量及其概率密度 \\
	满足以下条件的$f(x, y)$成为$(X, Y)$的概率密度
	\begin{equation}
		F(x, y) = \int_{-\infty}^{x} \int_{-\intfy}^{y}f(u, v)dudv, \quad -\infty <x, y < +\infty
	\end{equation}
	\item 边缘密度
	\begin{equation}
		f_X(x) = \int_{-\infty}^{+\infty}f(x, y)dy
	\end{equation}
	\begin{equation}
		f_Y(y) = \int_{-\infty}^{+\infty}f(x, y)dx
	\end{equation}
	\item 条件密度
	\begin{equation}
		f_{X|Y}(x|y) = \frac{f(x,y)}{F_Y(y)}, \quad f_Y(y) > 0
	\end{equation}
\end{enumerate}


\subsection{性质} % (fold)
\label{sub:性质}
\subsubsection{$F(x,y)$的性质}
\begin{enumerate}
	\item \\
	\begin{equation}
		F(-\infty, y) = F(x, -\infty) = F(-\infty, \infty) = 0
	\end{equation}
	\begin{equation}
		F(+\infty, +\infty) = 1
	\end{equation}

	\item \\
	\begin{equation}
		P(a<X\leq b, c<Y\leq d) = F(b,d) - F(b,c) - F(a,d) + F(a,c)
	\end{equation}
\end{enumerate}

\subsubsection{$P(X=x_i, Y=y_j)=p_{ij}$的性质}
\begin{enumerate}
	\item \\
	\begin{equation}
		p_{ij} \geq 0, \quad i,j = 1,2, \dots
	\end{equation}

	\item \\
	\begin{equation}
		\sum_i \sum_j p_{ij} = 1
	\end{equation}
\end{enumerate}

\subsubsection{$f(x,y)$的性质}
\begin{enumerate}
	\item \\
	\begin{equation}
		f(x, y) \geq 0
	\end{equation}
	\item \\
	\begin{equation}
		\int_{-\infty}^{+\infty} \int_{-\infty}^{+\infty}f(x,y)dxdy = 1
	\end{equation}
	\item \\
	\begin{equation}
		P((X,y)\in D) = \iint_Df(x,y) dxdy
	\end{equation}
	
\end{enumerate}

% subsection 性质 (end)
























		\section{随机变量的数字特征}
\subsection{随机变量的数学期望和方差}
\begin{enumerate}
	\item 数学期望
	\begin{enumerate}
		\item 离散型
			\begin{equation}
				E(X) = \sum_{k=1}^{+\infty}x_kp_k
			\end{equation}
			如果上式绝对收敛,则期望存在;否则期望不存在
		\item 连续型
			\begin{equation}
				E(X) = \int_{-\infty}^{+\infty}xf(x)dx
			\end{equation}
			同样地,若上式绝对收敛,则期望存在;否则不存在
	\end{enumerate}

	\item 数学期望的性质
	\begin{enumerate}
		\item 
		\begin{equation}
			E(CX) = CE(X)
		\end{equation}
		\item 
		\begin{equation}
			E(X \pm Y) = E(X) \pm E(Y)
		\end{equation}
		\item 若随机变量$X$, $Y$相互独立,则
		\begin{equation}
			E(XY) = E(X)E(Y)
		\end{equation}
		其实,$X, Y$不相关即可,不相关就是上式成立的充要条件
	\end{enumerate}

	\item $Y=g(X)$的数学期望
	\begin{enumerate}
		\item 离散型
			\begin{equation}
				E(Y) = E(g(X)) = \sum_{k=1}^{+\infty} g(x_k)p_k
			\end{equation}
			如果上式绝对收敛,则期望存在;否则期望不存在
		\item 连续型
			\begin{equation}
				E(Y) = E(g(X)) = \int_{-\infty}^{+\infty}g(x)f(x)dx
			\end{equation}
			同样地,若上式绝对收敛,则期望存在;否则不存在
	\end{enumerate}

	\item $Z=g(X,Y)$的数学期望
	\begin{enumerate}
		\item 离散型
			\begin{equation}
				E(Y) = E(g(X,Y)) = \sum_{i=1}^{+\infty}\sum_{j=1}^{+\infty} g(x_i, y_j)p_{ij}
			\end{equation}
			如果上式绝对收敛,则期望存在;否则期望不存在
		\item 连续型
			\begin{equation}
				E(Y) = E(g(X,Y)) = \int_{-\infty}^{+\infty}\int_{-\infty}^{+\infty}g(x,y)f(x,y)dxdy
			\end{equation}
			同样地,若上式绝对收敛,则期望存在;否则不存在
	\end{enumerate}

	\item 方差
	\begin{enumerate}
		\item 方差的定义
		\begin{equation}
			D(X) = E\{\left[X-E(X)\right]^2\}
		\end{equation}
		记做$\sigma^2$
		\item 方差计算公式
		\begin{equation}
			D(X) = E(X^2) - \left[ E(X) \right]^2
		\end{equation}
		另,因为方差恒$\geq 0$,故由上式可得出$E(X^2) \geq \left[ E(X) \right]^2$
	\end{enumerate}

	\item 方差的性质
	\begin{enumerate}
		\item 
		\begin{equation}
			D(aX+b) = a^2D(X)
		\end{equation}
		\item $X,Y$相互独立时
		\begin{equation}
			D(X\pm Y) = D(X) + D(Y)
		\end{equation}
		注意,不论是$X+Y$还是$X-Y$,最后的结果均是方差之和
	\end{enumerate}

	\item 常用随机变量的数学期望和方差
	\begin{enumerate}
		\item $0-1$分布
		\begin{align}
			E(X) &= p \\
			D(X) &= p(1-p)
		\end{align}
		\item 二项分布: $X \sim B(n,p)$
		\begin{align}
			E(X) &= np \\
			D(X) &= np(1-p)
		\end{align}
		\item 泊松分布: $X \sim P(\lambda)$
		\begin{align}
			E(X) &= \lambda \\
			D(X) &= \lambda
		\end{align}
		\item 几何分布: $P\{X=k\} = p(1-p)^{k-1}, \quad k = 1, 2, \dots, \quad 0<p<1$
		\begin{align}
			E(X) &= \frac{1}{p} \\
			D(X) &= \frac{1-p}{p^2}
		\end{align}
		\item 均匀分布: $X \sim U(a,b)$
		\begin{align}
			E(X) &= \frac{a+b}{2} \\
			D(X) &= \frac{(b-a)^2}{12}
		\end{align}
		\item 指数分布: $X \sim E(\lambda)$
		\begin{align}
			E(X) &= \frac{1}{\lambda} \\
			D(X) &= \frac{1}{\lambda^2}
		\end{align}
		\item 正态分布: $X \sim N(\mu, \sigma^2)$
		\begin{align}
			E(X) &= \mu \\
			D(X) &= \sigma^2
		\end{align}
	\end{enumerate}
\end{enumerate}





















		\section{大数定律与中心极限定律}
略
		\section{数理统计的基本概念}
\subsection{总体\&样本}
\begin{enumerate}
	\item 总体: 数理统计中所研究对象的某项指标$X$的全体称为总体。
	\item 样本: 如果$X_1, X_2, \dots, X_n$相互独立且都与总体$X$同分布,则称$X_1, X_2, \dots, X_n$为来自总体的简单随机样本,简称为样本。\\
	其中,$n$为样本容量,样本的具体观测值$x_1, x_2, \dots, x_n$为样本值,或称总体$X$的$n$个独立观测值。
	\begin{enumerate}
		\item 若总体$X$的分布为$F(x)$,则样本$X_1, X_2, \dots, X_n$的分布为
		\begin{equation}
			F_n(x_1, x_2, \dots, x_n) = \prod_{i=1}^{n}F(x_i)
		\end{equation}

		\item 若总体$X$的概率密度为$f(x)$,则样本$x_1, x_2, \dots, x_n$的概率密度为
		\begin{equation}
			f_n(x_1, x_2, \dots, x_n) = \prod_{i=1}^{n}f(x_i)
		\end{equation}
		\item 若总体$X$的概率分布为$P(X=a_j)=p_j, \quad j = 1,2, \dots$,则样本$X_1, X_2, \dots, X_n$的概率分布为
		\begin{equation}
			P(X_1=x_1, X_2=x_2, \dots, X_n=x_n) = \prod_{i=1}^{n}P(X_i = x_i)
		\end{equation}
		所以,在逻辑回归中,似然函数$L(\theta)$可以写成各个概率的乘积:
		\begin{equation}
			L(\theta)=\prod_{i=1}^{m}p(y^{(i)}|x^{(i)}; \theta)
		\end{equation}
		在机器学习中,每个数据点$(x^{(i)},y^{(i)})$就是总体中的一个个体(满足独立同分布)。所以上式中,乘积的上下限对应的是样本,故应为$i=1 \to m$
	\end{enumerate}
\end{enumerate}

\subsection{统计量}
\begin{enumerate}
	\item 统计量: 样本$X_1, X_2, \dots, X_n$的不含未知参数的函数$T = T(X_1, X_2, \dots, X_n)$称为统计量。
	\item 作为随机样本的函数,统计量本身也是一个随机变量。
	\item 若$x_1, x_2, \dots, x_n$是样本$X_1, X_2, \dots, X_n$的样本值,则数值$T(x_1, x_2, \dots, x_n)$是统计量$T = T(X_1, X_2, \dots, X_n)$的观测值。
	% \item 在迷你梯度下降中的Cost Function应该就是个统计量,$J(\theta)=\frac{1}{2}\sum_{i=1}^{m}\left[h_\theta(x^{(i)}) - y^{(i)}\right]^2$,其中,$\sum_{i=1}^{m}$就是这个$T${\color{red}{(正确性待查)}}
	\item 在指数分布族中,$T(y)$是个充分统计量
\end{enumerate}

\subsection{样本的数字特征}
设$X_1, X_2, \dots, X_n$是来自总体$X$的样本,则
\begin{enumerate}
	\item 样本均值
	\begin{equation}
		\bar X = \frac{1}{n}\sum_{i=1}^{n} X_i
	\end{equation}

	\item 样本方差
	\begin{equation}
		S^2 = \frac{1}{n-1}\sum_{i=1}^{n}(X_i - \bar X)^2
	\end{equation}
	样本标准差
	\begin{equation}
		S = \sqrt{\frac{1}{n-1}\sum_{i=1}^{n}(X_i - \bar X)^2}
	\end{equation}

	\item 样本$k$阶原点距
	\begin{equation}
		A_k = \frac{1}{n}\sum_{i=1}^{n}X_i^k, \quad k = 1, 2; \quad A_1 = \bar X
	\end{equation}

	\item 样本$k$阶中心距
	\begin{equation}
		B_k = \frac{1}{n}\sum_{i=1}^{n}(X_i - \bar X)^k, \quad k = 1, 2; \quad B_2 = \frac{n-1}{n}S^2 \neq S^2
	\end{equation}
\end{enumerate}

\subsection{样本的数字特征的性质}
\begin{enumerate}
	\item 若总体$X$具有数学期望$E(X)=\mu$,则
	\begin{equation}
		E(\bar X) = E(X) = \mu
	\end{equation}
	注意均值与期望的区别,均值针对的是样本(样本均值),期望针对的是事件(事件期望)。
	\item 若总体$X$具有方差$D(X)=\sigma^2$,则
	\begin{align}
		D(\bar X) &= \frac{1}{n}D(X) = \frac{\sigma^2}{n} \\
		E(S^2) &= D(X) = \sigma^2
	\end{align}
	\item 若总体$X$的$k$阶原点矩$E(X^k) = \mu_k, \quad k = 1, 2, \dots$存在,则当$n \to \infty$时:
	\begin{equation}
		\lim_{n\to \infty} \frac{1}{n}\sum_{i=1}^{n}X_i^k \xrightarrow{\quad P \quad} \mu_k, \quad k = 1, 2, \dots
	\end{equation}
\end{enumerate}

\subsection{常用统计抽样分布}
\subsubsection{$\chi^2$分布}
\begin{enumerate}
	\item $\chi^2$分布 \\
	若随机变量$X_1, X_2, \dots, X_n$相互独立且均服从标准正态分布$N(0,1)$,则称随机变量$\chi^2 = X_1^2+X_2^2+\dots+X_n^2$服从自由度为$n$的$\chi^2$分布,记做$\chi^2 \sim \chi^2(n)$
	\item $\chi^2$分布的性质 \\
	\begin{enumerate}
		\item $n$个相互独立的标准正态随机变量的平方和$\chi^2 = X_1^2+X_2^2+\dots+X_n^2$又称为$\chi^2(n)$的典型模式。
		\item 若$0<\alpha<1$,则称满足条件
		\begin{equation}
			P(\chi^2 > \chi_{\alpha}^{2}(n)) = \int_{\chi_{\alpha}^{2}(n)}^{+\infty}f(x)dx = \alpha
		\end{equation}
		的点为$\chi_{\alpha}^{2}(n)$为$\chi^2(n)$分布的上$\alpha$分位点。
		\begin{figure}[htbp]
			\centering
			\includegraphics[scale=0.9]{contents/分位点}
			\caption{$\chi^2$分布分位点示例图}
		\end{figure}
		\item $E(\chi^2) = n$,$D(\chi^2)=2n$
		\item 若$\chi_1^2 \sim \chi^2(n_1)$,$\chi_2^2 \sim \chi^2(n_2)$,且$\chi_1^2$与$\chi_2^2$相互独立,则$\chi_1^2 + \chi_2^2 \sim \chi^2(n_1+n_2)$
	\end{enumerate}
\end{enumerate}

\subsubsection{$t$分布}
\begin{enumerate}
	\item $t$分布 \\
	若随机变量$X$与$Y$相互独立,且$X\sim N(0,1)$,$Y\sim \chi^2(n)$,则称随机变量
	\begin{equation}
		T = \frac{X}{\sqrt{\frac{Y}{n}}}
	\end{equation}
	服从自由度为$n$的$t$分布,记做$T \sim t(n)$

	\item $t$分布的性质
	\begin{enumerate}
		\item 满足$X,Y$相互独立,$X\sim N(0,1)$,$Y\sim \chi^2(n)$三个条件的$T = \frac{X}{\sqrt{\frac{Y}{n}}}$称为$t(n)$的典型分布
		\item $t$分布的概率密度$f(x)$是偶函数,即$f(x) = f(-x)$
		\item 当$n$充分大时,$t(n)$分布近似于$N(0,1)$分布
		\item 若$0<\alpha<1$,则称满足条件
		\begin{equation}
			P(T > t_{\alpha}(n)) = \int_{t_{\alpha}(n)}^{+\infty}f(x)dx = \alpha
		\end{equation}
		的点为$t_{\alpha}(n)$为$t(n)$分布的上$\alpha$分位点。
		\item 因为$t(n)$分布的概率密度为偶函数,故可知$t$分布的双侧$\alpha$分位点$t_{\frac{\alpha}{2}}(n)$,即
		\begin{equation}
			P\left(|T|>t_{\frac{\alpha}{2}}(n)\right) = \alpha
		\end{equation}
		\begin{figure}[htbp]
			\centering
			\includegraphics[scale=0.9]{contents/t分布分位点}
			\caption{$t$分布分位点示例图}
		\end{figure}
		如图可知,$t_{1-\alpha}(n) = -t_{\alpha}(n)$(将$\frac{\alpha}{2}$用$\alpha$代替,公式更好看)。
	\end{enumerate}
\end{enumerate}

\subsubsection{$F$分布}
\begin{enumerate}
	\item $F$分布 \\
	若随机变量$X$与$Y$相互独立,且$X\sim \chi^2(n_1)$,$Y\sim \chi^2(n_2)$,则称随机变量
	\begin{equation}
		F = \frac{\frac{X}{n_1}}{\frac{Y}{n_2}}
	\end{equation}
	服从自由度为$(n_1, n_2)$的$F$分布,记做$F \sim F(n_1, n_2)$,其中,$n_1, n_2$分别称为第一自由度和第二自由度
	\item $F$分布的性质 \\
	\begin{enumerate}
		\item 满足$X,Y$独立,$X\sim \chi^2(n_1)$,$Y\sim \chi^2(n_2)$三个条件的$F=\frac{\frac{X}{n_1}}{\frac{Y}{n_2}}$称为$F(n_1, n_2)$的典型模式
		\item 若$0<\alpha<1$,则称满足条件
		\begin{equation}
			P(F > F_{\alpha}(n_1, n_2)) = \int_{F_{\alpha}(n_1, n_2)}^{+\infty}f(x)dx = \alpha
		\end{equation}
		的点为$F_{\alpha}(n_1, n_2)$为$F_{\alpha}(n_1, n_2)$分布的上$\alpha$分位点。
		\item 如果$F\sim F(n_1, n_2)$,则$\frac{1}{F}\sim F(n_2, n_1)$,且有
		\begin{equation}
			F_{1-\alpha}(n_1, n_2) = \frac{1}{F_{\alpha}(n_2, n_1)}
		\end{equation}
	\end{enumerate}
\end{enumerate}


\subsection{正态总体的抽样分布}
\subsubsection{一个正态总体的抽样分布}
设总体$X\sim N(\mu, \sigma^2)$,$X_1, X_2, \dots, X_n$是来自总体的样本,样本均值为$\bar X$,样本方差为$S^2$,则有
\begin{enumerate}
	\item $\bar X \sim N(\mu, \frac{\sigma^2}{n})$, $\frac{\bar X - \mu}{\frac{\sigma}{\sqrt{n}}} \sim N(0,1)$
	\item $\bar X$与$S^2$相互独立,且$\frac{(n-1)S^2}{\sigma^2} \sim \chi^2(n-1)$
	\item $\frac{\bar X - \mu}{\frac{S}{\sqrt{n}}} \sim t(n-1)$
	\item $\frac{1}{\sigma^2}\sum_{i=1}^{n}(X_i-\mu)^2 \sim \chi^2(n)$
\end{enumerate}

\subsubsection{两个正态总体的抽样分布}
设总体$X\sim N(\mu_1, \sigma_1^2)$和总体$Y \sim N(\mu_2, \sigma_2^2)$,$X_1, X_2, \dots, X_{n_1}$,$Y_1, Y_2, \dots, Y_{n_2}$是分别来自总体$X$和总体$Y$的样本,且相互独立,样本均值分别为$\bar X$和$\bar Y$,样本方差分别为$S_1^2$和$S_2^2$,则有
\begin{enumerate}
	\item $\bar X - \bar Y \sim N(\mu_1-\mu_2, \frac{\sigma_1^2}{n_1}+\frac{\sigma_2^2}{n_2})$, $\frac{(\bar X - \bar Y)-(\mu_1-\mu_2)}{\sqrt{\frac{\sigma_1^2}{n_1}+\frac{\sigma_2^2}{n_2}}} \sim N(0,1)$
	\item 若$\sigma_1^2 = \sigma_2^2$,则
	\begin{equation}
		\frac{\bar X - \bar Y-(\mu_1 - \mu_2)}{S_\omega \sqrt{\frac{1}{n_1} + \frac{1}{n_2}}} \sim t(n_1+n_2-2)
	\end{equation}
	其中,$S_\omega^2 = \frac{(n_1-1)S_1^2+ (n_2-1)S_2^2}{n_1+n_2-2}$
	\item 
	\begin{equation}
		\frac{\frac{S_1^2}{\sigma_1^2}}{\frac{S_2^2}{\sigma_2^2}} \sim F(n_1-1, n_2-1)
	\end{equation}
\end{enumerate}





















		\section{参数估计}
\subsection{点估计}
\begin{enumerate}
	\item 点估计 \\
	用样本$X_1, X_2, \dots, X_n$构造的统计量$\hat{\theta}(X_1, X_2, \dots, X_n)$来估计未知参数$\theta$称为点估计,统计量$\hat{\theta}(X_1, X_2, \dots, X_n)$称为估计量
	\item 无偏估计量 \\
	若$\hat{\theta}$是$\theta$的估计量,如果$E(\hat{\theta})=\theta$,则称$\hat{\theta} = \hat{\theta}(X_1, X_2, \dots, X_n)$是未知参数$\theta$的无偏估计量
	\item 更有效估计量 \\
	若$\hat{\theta_1}$和$\hat{\theta_2}$都是$\theta$的无偏估计量,且$D(\hat{\theta_1})\leq D(\hat{\theta_2})$,则称$\hat{\theta_1}$比$\hat{\theta_2}$更有效,或$\hat{\theta_1}$比$\hat{\theta_2}$更有效估计量
	\item 一致估计量 \\
	若$\hat{\theta}(X_1, X_2, \dots, X_n)$是$\theta$的估计值,如果$\hat{\theta}$依概率收敛于$\theta$,则称$\hat{\theta}(X_1, X_2, \dots, X_n)$为$\theta$的一致估计量
\end{enumerate}

\subsection{矩估计法}
\begin{enumerate}
	\item 矩估计法 \\
	用样本矩估计相应的总体矩,用样本矩的函数估计总体矩相应的函数,然后求出要估计的参数,称这种估计法为矩估计法
	\item 矩估计法的步骤 \\
	略
\end{enumerate}

\subsection{最大似然估计法}
\begin{enumerate}
	\item 似然函数
	\begin{enumerate}
		\item 离散型 \\
		假设其概率分布为:$P(X=a_i) = p(a_i;\theta), \quad i=1,2, \dots$,则似然函数为
		\begin{equation}
			L(\theta) = L(X_1, X_2, \dots, X_n; \theta) = \prod_{i=1}^{n}p(X_i;\theta)
		\end{equation}
		\item 连续型 \\
		假设其概率密度为$f(x;\theta)$,则其似然函数为:
		\begin{equation}
			L(\theta) = L(X_1, X_2, \dots, X_n; \theta) = \prod_{i=1}^{n}f(X_i;\theta)
		\end{equation}
	\end{enumerate}
	\item 最大似然估计法求解步骤
	\begin{enumerate}
		\item 写出似然函数
		\item 求似然函数的导数(或其对数的导数)
		\begin{equation}
			\frac{\mathrm{d}L(\theta)}{\mathrm{d}\theta} = 0
		\end{equation}
		或
		\begin{equation}
			\frac{\mathrm{d}\ln L(\theta)}{\mathrm{d}\theta} = 0
		\end{equation}
		\item 解方程组,得到其驻点
		\item 对每个驻点进行分析,得到最值
		\item 注意:有时使得似然函数达到最值的$\hat{\theta}$不一定是$L(\theta)$或$\ln L(\theta)$的驻点,这种情况下不能用似然方法来求解,应使用其他方法
	\end{enumerate}
\end{enumerate}

\subsection{区间估计}
\begin{enumerate}
	\item 置信区间 \\
	设$\theta$是总体$X$的未知参数,$X_1, X_2, \dots, X_n$是来自总体$X$的样本,对于给定的$\alpha(0<\alpha<1)$,如果两个统计量满足
	\begin{equation}
		P(\theta_1<\theta<\theta_2) = 1-\alpha
	\end{equation}
	则称随机区间$(\theta_1, \theta_2)$为参数$\theta$的置信水平(或置信度)为$1-\alpha$的置信区间(或区间估计),简称为$\theta$的$1-\alpha$置信区间,$\theta_1$和$\theta_2$分别称为置信下限和置信上限
	\item 一个正态总体参数的区间估计 \\
	略
	\item 两个正态总体参数的区间估计 \\
	略
\end{enumerate}
















		\section{假设检验}
\subsection{假设检验}
\begin{enumerate}
	\item 实际推断原理: 小概率事件在一次试验中是不会发生的。 \\
	实际推断原理又称小概率原理。
	\item 假设检验
	\begin{enumerate}
		\item 假设是关于总体的论断或命题,常用字母"$H$"表示。
		\item 假设分为基本假设$H_0$和备选假设 \\
		基本假设又称原假设、零假设 \\
		备选假设又称备择假设、对立假设
		\item 假设检验: 根据样本,按照一定规则判断所做假设$H_0$的真伪,并作出接受还是拒绝接受$H_0$的决定。
	\end{enumerate}
	\item 两类错误
	\begin{enumerate}
		\item 拒绝实际真的假设$H_0$(弃真)称为第一类错误
		\item 接受实际不真的假设$H_0$(纳伪)称为第二类错误
	\end{enumerate}
	\item 显著性检验
	\begin{enumerate}
		\item 显著性水平:在假设检验中允许犯第一类错误的概率称为显著水平,记为$\alpha(0<\alpha<1)$。
		它表现了对$H_0$弃真的控制程度,一般$\alpha$取0.1, 0.05, 0.01, 0.001等
		\item 显著性检验: 只控制第一类错误概率$\alpha$的统计检验称为显著性检验
		\item 显著性检验的一般步骤 \\
		略
	\end{enumerate}
\end{enumerate}


\subsection{正态总体参数的假设检验}
略











		\section{中心极限定理}
\subsection{待添加}
		\section{常用公式}
\begin{enumerate}
	\item 
	\begin{equation}
		\sum_{k=0}^{+\infty}\frac{x^k}{k!} \approx e^x
	\end{equation}

	\item 变上下积分求导公式
	\begin{equation}
		\frac{d}{dy}\int_{\alpha{(y)}}^{\beta{(y)}}f(x)dx = f(\beta(y)){\beta^{'}(y)} - f(\alpha(y)){\alpha^{'}(y)}
	\end{equation}

	\item 等比数列求和公式
	\begin{equation}
	 	\begin{aligned}
			S_n &= n a_1 \quad (q = 1) \\
			S_n &=  \frac{a_1(1-q^n)}{1-q} = \frac{a_1 - a_n q}{1-q} \quad (q \neq 1)
		\end{aligned}
	 \end{equation}

	 \item 最大最小值公式
	 \begin{align}
	 	\max(X,Y) &= \frac{X+Y+|X-Y|}{2} \\
	 	\min(X,Y) &= \frac{X+Y-|X-Y|}{2}
	 \end{align}


\end{enumerate}


\end{document}










